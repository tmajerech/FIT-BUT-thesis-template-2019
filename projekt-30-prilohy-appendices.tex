



\chapter{Instalační manuál}
\label{append:instalace}

\section{Lokální vývojové prostředí}
Stručný návod jak spustit lokální vývojové prostředí. (psáno na OS Windows)\\
Další informace je případně možno nalézt také v Cookiecutter-django dokumentaci\footnote{Cookiecutter-django dokumentace \url{https://cookiecutter-django.readthedocs.io/en/latest/developing-locally.html}}

\label{append:local}
\begin{enumerate}
    \item Nainstalujte prerekvizity\\
    Je třeba si nainstalovat všechny potřebné aplikace - python3, git, postgreSQL databázi a Mailhog\footnote{Mailhog: \url{https://github.com/mailhog/MailHog}}. 
    
    \item Vytvořte virtuální prostředí\\
    Je dobrým zvykem umisťovat python projekty zvlášť, každý do vlastního virtuálního prostředí, aby se vzájemně nemohly ovlivňovat. Přejděte do složky kde budete chtít mít projekt umístěn a otevřete powershell. Nyní proveďte příkaz
    
    \texttt{\$ python3 -m venv virtualenv}
    
    kterým je vytvořeno virtuální prostředí ve složce \texttt{/virtualenv}. 
    
    \item Aktivujte virtuální prostředí\\
    Nyní je třeba toto prostředí v konzoli aktivovat pomocí příkazu
    
    \texttt{\$ ./virtualenv/Scripts/activate}
    
    V tuto chvíli by měl být vidět název Vašeho virtuálního prostředí v konzoli. Odteď jsou implicitně všechny příkazy vykonávány uvnitř tohoto prostředí.
    
    \item Naklonujte Github\footnote{Github: \url{https://github.com/}} repozitář\\
    Jelikož tato práce bude trvale umístěna na mém veřejném Github repozitáři, lze ho použít k naklonování souborů projektu. Spuštěním příkazu 
    
    \texttt{\$ git clone https://github.com/tmajerech/volebni\_kalkulacka.git}
    
    se v aktuálním adresáři vytvoří podadresář \texttt{volebni\_kalkulacka}, sloužící jako kořenový adresář aplikace a obsahující všechny potřebné soubory.
    

    \item Nainstalujte potřebné python moduly\\
    Následující příkazy změní složku do kořenového adresáře aplikace a spustí instalaci python balíčků z instalačního seznamu. Díky tomu je není třeba instalovat po jednom ručně. Příkazem \texttt{upgrade pip} je také nejprve nainstalována poslední verze python instalátoru balíčků.

    \texttt{\$ python3 -m pip install –upgrade pip}\\
    \texttt{\$ cd volebni\_kalkulacka}\\
    \texttt{\$ pip install -r requirements/local.txt}

    Pokud aplikaci instalujete na operačním systému Linux, můžete v této fázi narazit na problémy s některými MYSQL balíčky. Následující kroky by mohly pomoct:
    \begin{itemize}
        \item \texttt{sudo apt-get install libmysqlclient-dev}
        \item Pokud po instalaci předchozí závislosti přetrvávají chybové hlášky jako například \\
        \texttt{No such file or directory \#include "mysql/udf\_registration\_types.h"},\\
        je možnost vyzkoušet smazání "\texttt{mysql/}"\ z hlavičkového souboru který tuto chybu způsobuje - cesta k němu by měla být viditelná v samotném chybovém hlášení. Tímto způsobem je možno odstranit i některé další chyby, ale toto řešení doporučuji zvážit pouze jako úplně poslední a dočasné.
    \end{itemize}
    
    \item Nastavte připojení k databázi\\
    V kořenovém adresáři aplikace (volebni\_kalkulacka) vytvořte soubor \texttt{.env} do kterého vložíte následující nastavení. Hodnoty upravte podle vlastních. Konstantu \texttt{USE\_DOCKER} ponechte beze změny.
    \begin{verbatim}
    POSTGRES_HOST=localhost
    POSTGRES_PORT=5432
    POSTGRES_DB=volebnikalkulacka
    POSTGRES_USER=postgres
    POSTGRES_PASSWORD=root
    USE_DOCKER=NO
    \end{verbatim}
    
    \item Vytvořte databázové tabulky\\
    Následující příkaz v kořenovém adresáři spustí tzv. \texttt{databázové migrace}, které vytvoří všechny potřebné tabulky podle vytvořených modelů. 

    \texttt{\$ python3 manage.py migrate}

    \item Spusťte lokální server\\
    Příkazem
    
    \texttt{\$ npm run dev}
    
    je spuštěn lokální vývojový server spolu s automatickým obnovením stránky při úpravě v HTML šablonách a také automatickou kompilací a doplněním CSS po uložení změn v SASS souboru - v tomto případě bez nutnosti obnovení stránky. Po startu serveru by se mělo automaticky otevřít okno s aplikací. V případě neotevření automaticky by se měla aplikace nacházet na url \url{http://127.0.0.1:8000/}.
    
    \item Spusťte Mailhog lokální emailový server\\
    Je třeba spustit stažený mailhog \texttt{.exe} soubor, který v příkazové řádce spustí lokální emailový server. Příkazový řádek je třeba nechat otevřený po dobu využívání emailů, jelikož nemá žádnou paměť a po vypnutí ztratí všechny zachycené emaily. Po zapnutí naleznete rozhraní emailového klienta na adrese \texttt{http://localhost:8025/}
    
    \item Vytvořte administračního uživatele.\\
    Například pro možnost mazat závadné komentáře je třeba vytvořit tzv. \texttt{super\\ uživatele}. K tomu slouží příkaz

    \texttt{\$ python3 manage.py createsuperuser}

    který už poté uživatele provede sám vytvářením.
    
    
\end{enumerate}




\section{Produkční prostředí}
\label{append:produkce}
Další informace je možné opět nalézt v dokumentaci pro Cookiecutter-django\footnote{Cookiecutter-django produkční nastavení: \url{https://cookiecutter-django.readthedocs.io/en/latest/deployment-with-docker.html}}

\begin{enumerate}
    \item Instalace prerekvizit\\
    Je třeba nainstalovat Python3, Git, Docker\footnote{Docker instalace: \url{https://docs.docker.com/get-docker/}}, Docker-compose\footnote{Docker-compose instalace: \url{https://docs.docker.com/compose/install/}}
    
    \item Naklonování repozitáře\\
    Ve vybraném adresáři proveďte příkaz
    
    \texttt{\$ git clone https://github.com/tmajerech/volebni\_kalkulacka.git}
    
    kterým naklonujete všechny soubory z githubu do nové složky \texttt{volebni\_kalkulacka} v aktuálním adresáři.
    
    \item Přidání enviromentálních proměnných pro django\\
    Po přechodu do kořenového adresáře \texttt{volebni\_kalkulacka} je třeba ve složce \texttt{.envs} vytvořit složku \texttt{.production} a v ní soubor \texttt{.django}. Do něj je vložte následující proměnné
    \begin{verbatim}
# General
# ----------------------------------------------------------------

DJANGO_SETTINGS_MODULE=config.settings.production
DJANGO_SECRET_KEY=
DJANGO_ADMIN_URL=admin
DJANGO_ALLOWED_HOSTS=[]

# Security
# -----------------------------------------------------------------
# TIP: better off using DNS, however, redirect is OK too
DJANGO_SECURE_SSL_REDIRECT=False

# Email
# -----------------------------------------------------------------
DJANGO_SERVER_EMAIL=
DJANGO_EMAIL_HOST_USER=
DJANGO_EMAIL_HOST_PASSWORD=

# django-allauth
# -----------------------------------------------------------------
DJANGO_ACCOUNT_ALLOW_REGISTRATION=True

# Gunicorn
# -----------------------------------------------------------------
WEB_CONCURRENCY=4

# Redis
# -----------------------------------------------------------------
REDIS_URL=redis://redis:6379/0
    \end{verbatim}
    
    
a doplňte všechny chybějící proměnné. Do \texttt{DJANGO\_ALLOWED\_HOSTS} přijde název produkční domény, případně také její IP adresa a další povolené a url. Nastavení emailu je možné simulovat například přes osobní gmail účet s povoleným připojením třetích stran.

\item Přidání enviromentálních proměnných pro postgreSQL\\
Opět ve složce \texttt{.production} je třeba vytvořit soubor \texttt{.postgres} a do něj vložit

\begin{verbatim}
# PostgreSQL
# ------------------------------------------------------------------
POSTGRES_HOST=postgres
POSTGRES_PORT=5432
POSTGRES_DB=volebni_kalkulacka
POSTGRES_USER=vjZtIkTyKgMaZGLVWHVLySsLllRCbGVx
POSTGRES_PASSWORD=
    LGpkCqeeUrvzvonMonlgeSa11EBqxOM8IAZpyzNuDb7exKxrIVZZLqkYDGTtH15s
\end{verbatim}

kde text za proměnnou \texttt{POSTGRES\_PASSWORD} bude na jednom řádku, zde je na samostatném kvůli šířce stránky. Tyto údaje byly vygenerovány při vytváření šablony aplikace v Cookiecutter-django a mělo by je jít kdykoliv nahradit novými údaji korespondujícími s nastavením postgreSQL kontejneru.


\item Nastavení SSL\\
SSL, neboli \texttt{Secure Sockets Layer} (vrstva bezpečných socketů) je bezpečnostní standard pro vytváření šifrovaného spojení mezi serverem a klientem, v našem případě konkrétně mezi webovým serverem (stránkou) a prohlížečem. Nemít HTTPS znamená, že je možné z komunikace odposlechnout například přihlašovací údaje \cite{cookiecutter-ssl}.
\\

V této aplikaci SSL zajišťuje služba Traefik. Mně se bohužel nepodařilo tuto službu spustit, protože jsem ji odkládal až úplně nakonec, ale nemělo by to být složité. Postupujte podle Cookiecutter-django manuálu\footnote{Cookiecutter-django SSL manuál: \url{https://cookiecutter-django.readthedocs.io/en/latest/deployment-with-docker.html}}

\item Sestavení docker kontejnerů\\
V kořenovém adresáři aplikace spusťte příkaz

\texttt{\$ docker-compose -f production.yml build}

kterým se sestaví všechny potřebné docker kontejnery. Příkazem

\texttt{\$ docker ps -a}

lze zobrazit všechny vytvořené kontejnery včetně neaktivních.

\item Spuštění aplikace\\
Opět v kořenovém adresáři spusťte příkaz

\texttt{\$ docker-compose -f production.yml up}

kterým by se měly spustit všechny potřebné docker kontejnery a aplikace by měla sama začít poslouchat na portu 80 pro http a 443 pro https.

\item Naimportujte data\\
Spusťte příkazem 

\texttt{\$ docker-compose -f production.yml\\ 
        run --rm django python manage.py initial\_import }

import dat do databáze. Poté nastavte cron, který bude tento příkaz spouštět pravidelně v noci. Tato funkce je náročná na výkon serveru, takže doporučuji spouštět když nebude server zatížený. Vzor nastavení cronu můžete převzít z kapitoly \ref{import_dat}.
    
\end{enumerate}

Pokud tuto práci někdo bude dále rozšiřovat a narazí v průběhu na nečekané komplikace, může mě kontaktovat a jsem ochoten, pokud to bude v mých silách, pomoct s nastavením a pochopením.

