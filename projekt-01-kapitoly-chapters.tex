
% Autor: Tomáš Majerech
\chapter{Úvod}
Každé čtyři roky stojí obyvatelé České republiky před rozhodnutím, komu dají svůj hlas ve volbách do Poslanecké sněmovny. Velké množství z~nich nemá mnohdy čas ani chuť pravidelně sledovat dění na politické scéně a nejsou si tudíž jistí, kdo z~kandidátů či stran by dostatečně dobře zastupoval jejich názory a zájmy. Proto stále častěji lidé vyhledávají volební kalkulačky, kde můžou pomocí odpovědí na pár otázek zjistit shodu s~některým z~kandidátů nebo s~některou ze stran. Na tomto již existujícím principu staví také tato bakalářská práce a dále ho rozšiřuje a modernizuje. 
\par Skutečnost, že lidé volební kalkulačky pravidelně využívají, skýtá prostor k~potenciálnímu budoucímu využítí této práce širokou veřejností a její existence by tedy nekončila ve školním archivu, nýbrž by se mohla stát univerzálním nástrojem, usnadňujícím život řadě běžných lidí. To byl také jeden z~důvodů, proč mi toto téma dávalo smysl.
\par Ve druhé kapitole (\ref{chap:Prerekvizity}) je čtenář seznámen s~problematikou legislativního procesu v~České republice, strukturou Poslanecké sněmovny a základními pojmy nutnými k~porozumění této práci. Také je zde popsána existující konkurence (v~době psaní této práce) a zdroj dat. Třetí kapitola (\ref{chap:technologie}) rozebírá použité technologie. Kromě HTML a CSS, které se nachází prakticky na každém webu, jsem se rozhodl pro tuto práci využít na klientské části framework\footnote{Framework: aplikační rámec} Bootstrap a css preprocessor SASS. Na serverové(\texttt{backend}) části je využit python framework Django, k~uložení dat je využita objektově-relační databáze PostgreSQL s~podporou technologie Redis, zajišťující redukci počtu nutných databázových výpočtů pomocí jejich ukládání do mezipaměti. Na serveru je celá aplikace nasazena v~Docker kontejnerech.  Ve čtvrté kapitole (\ref{chap:navrh}) je popsán postup návrhu webového rozhraní a struktury uložených dat, následovaná kapitolou implementace (\ref{chap:implementace}), rozebírající nejdůležitější části vývoje. Výčet kapitol uzavírá kapitola číslo \ref{chap:testovani}, shrnující uživatelské testování a závěr práce (\ref{chap:zaver}).



\chapter{Prerekvizity}
\label{chap:Prerekvizity}
V~úvodu této kapitoly jsou stručně shrnuty základní pojmy a procesy týkající se zaměření této práce, zejména z~oblasti legislativního procesu. V~další části jsou zmíněny již existující konkurenční řešení a jejich nedostatky. Následuje popis primárního zdroje dat.


\section{Základní pojmy}

\subsection{Poslanecká sněmovna}
Poslanecká sněmovna je jedna ze dvou složek, tzv. komor, Parlamentu České republiky. Má 200 členů, tzv. poslanců, volených v~přímé volbě na 4 roky \cite{ustava-parlament}.  \\
Smyslem Poslanecké sněmovny je tvorba a přijímání zákonů. Návrhy zákonů mohou přijít od poslanců, jejich skupin, krajských zastupitelstev, senátu a vlády \cite{ustava-navrh_zakona}.  \\
Schvalování zákonů má 3 části zvané čtení.
\begin{enumerate}
    \item čtení: \\
    Rozhoduje se o~tom, jestli se PS bude návrhem vůbec zabývat. Pokud projde, dochází k~dalšímu hlasování o~tom, jakému výboru bude návrh přiřazen k~přezkoumání.
    V~některých případech může být návrh schválen PS už v~této fázi, pokud o~to navrhovatel požádá.
    \item čtení: \\
    Na začátku vystoupí jeden zpravodaj z~každého přiřazeného výboru a informuje PS zprávou a doporučením, na kterém se výbor usnesl. Také se přednáší pozměňovací návrhy.
    \item čtení: \\
    Hlasuje se o~pozměňovacích návrzích ze 2. čtení, případně o~jiných návrzích. Následuje hlasování o~přijetí zákona jako celku.\\
    Po přijetí pokračuje návrh do senátu, což je druhá komora Parlamentu České republiky. Ten může návrh odmítnout a vrátit zpět do PS. V~takovém případě musí PS o~návrhu opět hlasovat a pokud je návrh schválen nadpoloviční většinou všech poslanců (101 hlasů), je senát přehlasován. Posledním krokem je podpis prezidenta České republiky. Ten má ale opět právo návrh vetovat. Zde je postup stejný jako u~veta senátního a PS ho může opět přehlasovat nadpoloviční většinou všech poslanců.
\end{enumerate}
Zjednodušený průběh legislativního procesu je znázorněn na obrázku \ref{fig:legislativni_proces}.

\begin{figure}
    \centering
    \includegraphics[width=0.9\textwidth]{obrazky-figures/legislativni_proces_my.pdf}
    \caption{Zjednodušený graf legislativního procesu v~Poslanecké sněmovně}
    \label{fig:legislativni_proces}
\end{figure}
 
\subsection{Výbory poslanecké sněmovny}
\label{vybory}
\par Výbory poslanecké sněmovny jsou skupiny lidí zaměřené na určitý odborný problém, složené jak z~členů vlády, tak opozice. Tyto skupiny se návrhem zákona zabývají v~odborné rovině a následně na 2. čtení PS doporučí návrh přijmout, zamítnout, nebo navrhnou změny.\\
Jeden návrh může být přiřazen více výborům, každý z~výborů má svého zpravodaje, ale jen jeden z~nich je tzv. garanční - ručí za přidělený návrh. \\

\subsection{Jednání PS}
\par Jednání, na kterých jsou návrhy zákonů projednávány, se nazývají schůze Poslanecké sněmovny. Aby byla sněmovna usnášeníschopná, musí být přítomno alespoň třetinové kvórum, což odpovídá 67 poslancům \cite{ustava-kvorum}. \\
Průběh jednání z~hlediska hlasování vypadá následovně:
\begin{itemize}
    \item Přihlášení účastníků
    \item Určení zapisovatelů a hlasování o~jejich schválení
    \item Hlasování o~jednotlivých bodech aktuální schůze \\
    - v~případě, že vůbec dojde k~jejich projednání
    \item Hlasování o~pořadí projednávaných bodů
    \item Hlasování o~konkrétních bodech na pořadu\\
    - může dojít ke zpochybnění hlasování a k~jeho opakování
    \item V~průběhu schůze může dojít také k~některým dalším, méně zajímavým, hlasovnáním například o~odložení projednávání nějakého aktuálního bodu na později, či vyhlášení přestávky.
    
\end{itemize}

\subsection{Kvórum}
Kvórum je minimální počet hlasů nutný pro přijetí návrhu hlasováním. Běžně je kvórum nadpoloviční většina přítomných poslanců, v~některých případech se však může lišit. Například kvórum nadpoloviční většiny všech poslanců nutné pro přehlasování veta Senátu, či prezidenta, nebo kvórum 3/5 přítomných nutné pro přijetí ústavních zákonů \cite{zakon-kvorum}.

\subsection{Sněmovní tisky}
Sněmovní tisky jsou, zjednodušeně řečeno, dokumenty předkládané poslancům jako podklady pro jednotlivé body programu. 

\section{Existující řešení}
Na internetu lze v~době psaní této práce nalézt více verzí českých i zahraničních volebních kalkulaček. Jelikož jsou volby obecně pro jednotlivé země silně specifické, zahraniční aplikace zde nebudu rozebírat. Z~těch českých jsou nejzajímavější hlavně volebnikalkulacka.cz\footnote{volebnikalkulacka.cz: \url{https://volebnikalkulacka.cz/}} a ta na zpravodajském portálu iDnes.cz\footnote{Volební kalkulačka na idnes.cz: \url{https://volby.idnes.cz/volebni-kalkulacka-kraje.aspx}}. Za zmínku stojí také Postvolítko\footnote{Postvolítko: \url{https://www.fi.muni.cz/~xzahrad4/Hlasovatko/index.cgi?o=8}}, projekt studentů Masarykovy univerzity. 

\subsection{volebnikalkulacka.cz}
Princip této volební kalkulačky  je jednoduchý. Autor, či autoři, vytvoří soubor otázek, který by podle jejich názoru mohl uživatele zajímat. Tyto otázky jsou následně rozeslány všem zúčastněným politickým stranám, či kandidátům. Na základě matematických modelů jsou poté odpovědi zpracovány a vybrány nejlepší otázky. Tyto jsou nakonec prezentovány v~samotné kalkulačce (obr. \ref{fig:volebnikalkulacka-otazka}), kde na ně odpovídá uživatel. Po ukončení dotazníku je uživateli představena jeho procentuální shoda v~odpovědích s~jednotlivými stranami či kandidáty (obr. \ref{fig:volebnikalkulacka-vysledky}) \cite{volebnikalkulacka-info}.
\par Výhoda tohoto systému je v~jeho jednoduchosti a přehlednosti. Otázky bývají jednoznačné, srozumitelné a týkají se aktuálních témat, takže běžný uživatel nemusí mít nic než obecný přehled o~hlavním dění v~České republice a ve svém kraji. 
\par Za jeho nevýhodu by se dala považovat závislost na pravdivých odpovědích a nepopulistickém přístupu. Může se totiž stát, že kandidát vyplní dotazník podle toho co zrovna v~tu chvíli jeho potenciální voliči požadují a následně ve skutečných hlasováních v~Poslanecké sněmovně hlasovat jinak.
\\

\begin{figure}
    \centering
    \includegraphics[width=1\textwidth]{obrazky-figures/volebnikalkulackacz-uvod.png}
    \caption{Úvodní stránka aplikace volebnikalkulacka.cz}
    \label{fig:volebnikalkulacka-uvod}
\end{figure}

\begin{figure}
    \centering
    \includegraphics[width=1\textwidth]{obrazky-figures/volebnikalkulacka-otazka.png}
    \caption{Příklad otázky v~aplikaci volebnikalkulacka.cz}
    \label{fig:volebnikalkulacka-otazka}
\end{figure}

\begin{figure}
    \centering
    \includegraphics[width=1\textwidth]{obrazky-figures/volebnikalkulacka-vysledky.png}
    \caption{Příklad prezentace výsledků uživateli v~aplikaci volebnikalkulacka.cz}
    \label{fig:volebnikalkulacka-vysledky}
\end{figure}




\subsection{iDnes.cz}
Princip této aplikace je totožný jako u~volebnikalkulacka.cz. Liší se pouze některé otázky a grafický vzhled, který je zde přizpůsoben celkovému stylu zpravodajského webu idnes.cz\footnote{idnes.cz: \url{https://www.idnes.cz/}} , viz obrázek \ref{fig:idnes-uvod}. Dokonce mají obě aplikace i stejného autora, kterým je sdružení \mbox{kohovolit.eu}\footnote{kohovolit.eu: \url{http://kohovolit.eu/}}.

\begin{figure}
    \centering
    \includegraphics[width=0.7\textwidth]{obrazky-figures/idnes-uvod.png}
    \caption{Úvodní strana volební kalkulačky na webu Idnes.cz}
    \label{fig:idnes-uvod}
\end{figure}

\subsection{Postvolítko}
Přestože nejde o~volební kalkulačku, je tato studentská aplikace velmi zajímavý nástroj pro zkoumání dat z~Poslanecké sněmovny. Ta jsou zde přehledně a srozumitelně rozřazena do 3 různých přehledů - hlasování, poslanci a strany, jak můžeme vidět na obrázku \ref{fig:postvolitko-uvod}.

\begin{figure}
    \centering
    \includegraphics[width=0.7\textwidth]{obrazky-figures/postvolitko-uvod.png}
    \caption{Úvodní strana webové aplikace Postvolítko}
    \label{fig:postvolitko-uvod}
\end{figure}


\par Přehled hlasování nabízí výčet všech poslaneckých schůzí ve zvoleném volebním období. Tyto schůze jsou dále doplněny o~přehled jednacích bodů schůze a k~nim náležejících hlasování, obr. \ref{fig:postvolitko-hlasovani}. \\
U~jednotlivých hlasování lze dále otevřít detail s~konkrétními jmény poslanců a jejich volbou.

\begin{figure}
    \centering
    \includegraphics[width=0.7\textwidth]{obrazky-figures/postvolitko-hlasovani.png}
    \caption{Přehled schůzí a hlasování PS v~aplikaci Postvolítko}
    \label{fig:postvolitko-hlasovani}
\end{figure}

\par Přehled poslanců nabízí seznam všech aktivních poslanců ve zvoleném volebním období. Pod jménem poslance se dále lze dostat na detail jeho osoby, kde nalezneme kromě jeho veřejně dostupných osobních údajů také statistiky jeho hlasování a absencí.


\par V~přehledu stran nalezneme seznam všech aktuálně zúčastněnch stran v~poslanecké sněmovně. Přes jednotlivé strany se dá opět dostat na jejich detail. Zde je opět seznam poslanců, pouze omezen na členy konkrétní strany.

\par Dále zde nalezneme žebříčky podle různých statistik, například podle počtu hlasování pro, počtu neomluvených hodin, či nejnižší absence. 
\par Poslední funkcí Postvolítka je porovnávání dvou vybraných poslanců a zobrazení jejich procentuální shody, vycházející ze stejných voleb u~jednotlivých hlasování.


\section{Výchozí data}
Data, ze kterých tato práce čerpá, se nachází na oficiálním webu Poslanecké sněmovny Parlamentu České republiky\footnote{Oficiální web Poslanecké sněmovny: \url{https://www.psp.cz/sqw/hp.sqw}}. Konkrétně se jedná o~data agend Poslanecké sněmovny\footnote{Data agend Poslanecké sněmovny: \url{https://www.psp.cz/sqw/hp.sqw?k=1300}}, poskytovaná veřejnosti zdarma pod podmínkou uvedení zdroje a data zpracování \cite{pspData}.
\par Zde dostupné záznamy popisují všechny schůze Poslanecké sněmovny až do roku 1993. Ty jsou dále děleny na popis samotné schůze a jejího stavu, jednotlivé body pořadu, jejich popis, stav a výsledek. Kromě samotných schůzí jsou k~dispozici také záznamy o~všech poslancích v~tomto volebním období - fotky, pracovní emaily a telefony, ostatní funkce které vykonávají, či jejich hlasování a absence.
\par Data jsou strukturována v~tabulkách, mezi nimiž jsou vzájemné vazby. Na obrázku \ref{fig:psp-tabulka-priklad} je struktura jedné z~databázových tabulek. Ty nejpodstatnější vazby mezi tabulkami pro tuto práci jsou naznačeny na obrázku \ref{fig:psp-data-diagram}.
\par Jednou denně jsou všechna data aktualizována a doplněna o~nové záznamy. Tato aktualizace probíhá v~noci.



\begin{figure}
    \centering
    \includegraphics[width=0.8\textwidth]{obrazky-figures/psp-tabulka-priklad.png}
    \caption{Popis struktury jedné z~datových tabulek Poslanecké sněmovny\cite{pspData}}
    \label{fig:psp-tabulka-priklad}
\end{figure}

\begin{figure}
    \centering
    \includegraphics[width=1\textwidth]{obrazky-figures/data_diagramy.pdf}
    \caption{Diagram znázorňující vztahy některých datových tabulek}
    \label{fig:psp-data-diagram}
\end{figure}




%%%%%%%%%%%%%%%%%%%%%%%%%% TECHNOLOGIE  %%%%%%%%%%%%%%%%%%%%%%%%%%%%%%%
\chapter{Použité technologie}
\label{chap:technologie}
V~této kapitole jsou stručně popsány použité technologie. Jedná se o~webovou stránku, proto jsou na úvod představeny základní techniky k~vytvoření statického webu. Po nich následuje představení technologií zajišťujících dynamičnost. Zde patří python a jeho framework Django. Pro nasazení na produkčním serveru a mezipaměť (anglicky \texttt{cache}) jsou využity technologie Docker\footnote{Docker: \url{https://www.docker.com/}} a Redis\footnote{Redis: \url{https://redis.io/}}, které propojuje s~vnějším světem technologie Traefik\footnote{Traefik: \url{https://doc.traefik.io/traefik/}}.
\par Kapitola je zakončena popisem použitého databázového systému PostgreSQL.


\section{HTML, CSS}
HTML je zkratka pro \texttt{hyper text markup language}. Jak už název napovídá, tento jazyk není programovací, nýbrž strukturovací (značkovací). To znamená, že pomocí něj dokážeme vytvořit strukturu a význam částí dokumentu, který následně dokáží správně zobrazit a~interpretovat webové prohlížeče. Často se přirovnává ke kostře lidského těla \cite{HTML}.  

\par CSS je zkratkou pro \texttt{cascading style sheets} a definuje jak budou HTML elementy zobrazeny. V~překladu název znamená \uv{kaskádové styly}, kdy kaskádě odpovídá vrstvení jednotlivých stylů na sebe, ovšem vždy platí ten poslední, případně ten nejspecifičtější. Při použití předchozího přirovnání by CSS bylo jako maso a kůže \cite{CSS}. 


\section{Bootstrap}
Bootstrap\footnote{Bootstrap: \url{https://getbootstrap.com/}} je aktuálně nejpoužívanější CSS framework pro vývoj responsivních a tzv. \texttt{mobile-first} stránek \cite{BOOTSTRAP}. Poskytuje předpřipravené HTML a CSS komponenty, pomocí kterých se dá rychle a jednoduše vytvořit design webu. Klíčovým aspektem je využívání tříd, kterými je vývojář schopen jednoduše přiřadit HTML elementům styly, které bootstrap již v~základu poskytuje, aniž by při tom musel neustále přecházet mezi HTML šablonou a CSS dokumentem. Jednou z~nejvyužívanějších funkcí bootstrapu je možnost virtuálního rozdělení obsahu elementu na 12 stejných sloupců, se kterými lze pak jednoduše pracovat pomocí přidávání tříd na elementy obsahu. Velkou výhodou je také jednoduchá práce s~responsivitou, opět pomocí různých tříd.
\par  Obecně existují dva způsoby myšlení při vytváření webové stránky. První je navrhnout nejprve verzi pro stolní počítače, respespektive celkově velké obrazovky a pak ji postupně obírat o~funkcionalitu, aby byla použitelná i na telefonu. Tento způsob byl přijatelný dříve, kdy mobilní telefony sloužily primárně k~telefonování a počet návštěvníků webu z~nich nebyl nijak závratný. Se zvyšujícím se využitím mobilních telefonů pro prohlížení internetu, kdy v~dnešní době více než polovina lidí tzv. \uv{surfuje} na telefonu \cite{WEB-DEVICE-USAGE}, je však již považován za nedostatečný.
\par Tím druhým je právě \texttt{mobile first}, kdy se nejprve navrhne design pro mobilní telefony a až s~přibývajícím místem na obrazovce se doplňují i pokročilejší designové prvky a funkce. Tento způsob je v~současné době obecně využívanější, jelikož je pro vývojáře ve většině případů mnohem komfortnější a zaměřuje se primárně na jednoduchost a ovladatelnost právě na malých obrazovkách, kde to mnohdy může být pro uživatele problém.

\section{Sass}
Sass\footnote{Sass: \url{https://sass-lang.com/}} je jeden z~několika běžně používaných CSS preprocesorů\footnote{CSS preprocesory: \url{https://raygun.com/blog/css-preprocessors-examples/}}. Ve své podstatě je CSS preprocesor nástroj, respektive jazyk, který zjednodušuje, zpřehledňuje a zrychluje aplikaci CSS webových stránek. Snaží se zvýšit efektivitu psaní kódu pomocí minimalizace nutnosti kopírovat velké bloky již jednou napsaného kódu, což je konkrétně u~CSS běžná věc. Přidává například možnost použití proměnných, zanořování selektorů, využívání mixinů. 
\par Prohlížeče nicméně stále ještě pracují pouze s~čistým CSS, proto se do něj jazyk preprocesorů musí překládat před tím, než ho server odešle uživateli. 


\section{JavaScript}
JavaScript\footnote{JavaScript: \url{https://developer.mozilla.org/en-US/docs/Web/JavaScript}} je scriptovací jazyk umožňující dynamicky měnit obsah stránky bez nutnosti stránku znovu načíst. Kromě toho umí také spoustu dalších věcí, kupříkladu ovládat multimédia a animovat jednotlivé elementy. Naprostá většina moderních prohlížečů má jeho podporu již vestavěnou, mezi výjimky patří například textový prohlížeč Lynx\footnote{Lynx: \url{https://lynx.browser.org/}}. Podle dostupných statistik \cite{JAVASCRIPT-USAGE} je dnes JavaScript využíván na 95\% webových stránek.

\section{jQuery}
jQuery\footnote{jQuery: \url{https://jquery.com/}} je malá knihovna určená ke zjednodušení práce s~JavaScriptem. Hlavní snahou je zmenšení množství napsaného kódu, nutného k~vytvoření požadované funkcionality, spolu se zjednodušením používání některých JavaScriptových prvků, kupříkladu AJAX\footnote{AJAX: Asynchronní JavaScript a XML} volání \cite{JQUERY}. Další z~výhod jQuery je obrovské množství dostupných pluginů, díky kterému je často možné jednoduše odstranit i komplikované problémy pomocí již hotového řešení\footnote{jQuery pluginy: \url{https://plugins.jquery.com/chosen/}}.  
\par jQuery je v~současné době nejspíše nejpopulárnější JavaScriptový framework, využívaný na více než 19 milionech webových stránek \cite{JQUERY-USAGE}. Najdeme ho obsažen také v~populárním systému správy obsahu Wordpress\footnote{Wordpress: \url{https://wordpress.org/}}, nicméně postupně se objevuje snaha od jeho použití upouštět kvůli zbytečnému zvyšování velikosti stránky.

\section{Python}
Python\footnote{Python: \url{https://www.python.org/}} je interpretovaný, objektově orientovaný, vysokoúrovňový programovací jazyk s~dynamickou sémantikou \cite{PYTHON}. \\
Kromě toho je v~současné době asi nejrychleji rostoucím a nejoblíbenějším programovacím jazykem napříč celou škálou oborů, díky své jednoduchosti a intuitivnosti. Využívá se ve všech možných aplikacích, od základní automatizace opakujících se činností, například zpracování excelu, nebo zpracování dat z~webových stránek, přes matematické operace, analýzu a vizualizaci dat, až po vytváření umělé inteligence. Mimo to ho lze využít také k~vytváření aplikací - mobilních, desktopových i webových. Právě kvůli možnosti vytvoření webové aplikace, která bude zároveň mít možnost zpracovávat a analyzovat data bez nutnosti přidávání dalších nástrojů, byl jazyk python zvolen pro serverovou část této práce.

\section{Django}
Django\footnote{Django: \url{https://www.djangoproject.com/}} je volně dostupný, open-source framework pro webové aplikace napsané v~jazyce Python. Jeho smyslem je ulehčit a urychlit vytváření webových aplikací. Za tímto účelem poskytuje již předpřipravenou strukturu, zajišťující například zabezpečení, práci s~databází, šablonování, autentizaci a další. Django vychází z~návrhového vzoru MVC, což je jeden z~nejpopulárnějších návrhových vzorů využívaných pro webové stránky, kde \texttt{Controller} je nahrazen šablonami (anglicky \uv{Template}), tedy \texttt{MVT}. Kromě toho Django obsahuje množství volně dostupných zásuvných modulů (\texttt{pluginů}) rozšiřujících jeho funkcionalitu. Nechybí také přehledná a rozsáhlá dokumentace a široká uživatelská základna nabízející pomoc při řešení problémů.

\subsubsection{MVT}
MVT, čili \texttt{model-view-template} je návrhový vzor rozdělující aplikaci na 3 základní části. Model je databázová část, starající se o~to jak jsou data uchovávána, o~jejich ukládání a čtení. View obsahuje logiku a komunikuje s~Modelem. Template je HTML šablona, které View předá data a kterou django nakonec zobrazí (vyrenderuje) uživateli. Diagram fungování MVT je znázorněn na obrázku \ref{fig:MVT}.

\begin{figure}
    \centering
    \includegraphics[width=0.8\textwidth]{obrazky-figures/MVT.pdf}
    \caption{Fungování návrhového vzoru MVT}
    \label{fig:MVT}
\end{figure}

\section{Cookiecutter}
Cookiecutter-Django\footnote{Cookiecutter: \url{https://github.com/pydanny/cookiecutter-django}} je generátor základní kostry Django webové aplikace. Uživatel si při spuštění generátoru postupně vybírá jaké technologie chce ve svém projektu použít a aplikace mu podle toho sestaví plně funkční základní moduly a soubory. Díky tomu si projekt udržuje určitý vhodný systém a uspořádání už od počátku.

\section{Docker}
Docker je, zjednodušeně řečeno, virtualizační nástroj umožňující umístit aplikaci spolu se všemi jejími potřebami na server nezávisle na jeho druhu a nainstalovaných aplikacích. Docker kontejner poskytuje základní, předpřipravené prostředí, ve kterém aplikace následně běží, podobně jako běžné virtualizované systémy. Jeho výhodou však je, že na rozdíl od nich není tak náročný, je menší a jednodušeji spravovatelný.  

\par Každá služba má svůj vlastní Docker kontejner - jeden kontejner zvlášť pro každou ze služeb Redis, PostgreSQL, Django a Traefik. Tyto mezi sebou můžou komunikovat díky propojení vlastní lokální sítí. 

\par 
Aby nebylo potřeba ručně při každém zapnutí aplikace konfigurovat každý z~kontejnerů zvlášť, existuje \texttt{docker-compose}. Správce aplikace sepíše všechno nastavení do souboru zvaného \texttt{Dockerfile} a docker-compose se po spuštění postará o~zbytek.

\section{Redis}
Redis je NoSQL\footnote{NoSQL: \url{https://aws.amazon.com/nosql/}} databáze. Běžné SQL databáze využívají k~ukládání dat tabulky, které jsou dále členěny na sloupce a řádky, kdy sloupce mají pevně daný typ dat které mohou obsahovat. Redis využívá dvojice klíč:hodnota, kde hodnota může nabývat jeden z~několika typů. Je schopen celý pracovat uvnitř paměti, což z~něj činí velmi rychlý a efektivní nástroj vhodný pro použití jako cache, kdy díky němu můžeme místo zatěžování serveru opakovanými náročnými operacemi pouze vyčíst již jednou získané výsledky z~mezipaměti s~velmi nízkými náklady. V~současné době je tímto způsobem využíván například na Twitteru, StackOverflow či Pinterestu.

\section{Traefik}
Traefik je open source proxy. Jeho úlohou je v~tomto případě mapování vnějších portů na aplikaci. Tzn. pokud někdo přistoupí na adresu serveru na port 80, Traefik tento přístup přepošle (přeloží) na port v~django docker kontejneru. Konkrétní port závisí na nastavení v~Dockerfile.


\section{PostgreSQL}
PostgreSQL\footnote{PostgreSQL: \url{https://www.postgresql.org/}} je výkonný, objektově-relační, open-source, databázový systém využívající a~zároveň rozšiřující jazyk SQL. Jeho vznik se datuje až do roku 1986 a za dobu své existence si získal pověst spolehlivého, robustního a jednoduše rozšiřitelného řešení databáze \cite{PostgreSQL}.
\par Při volbě databázového systému byl původní kandidát MySQL, jelikož s~ním často pracuji a všechno potřebné jsem tak měl již připravené. Nicméně podle některých názorů \cite{WHY-POSTGRES1}\cite{WHY-POSTGRES2} je obecně lepší využívat databázový systém PostgreSQL, jelikož je na něj Django lépe připravené a používají ho i sami vývojáři, takže bude lépe dokumentován.

\section{Další technologie}
Za zmínku stojí také \texttt{npm}\footnote{NPM: \url{https://www.npmjs.com/}}, což je nejpoužívanější správce balíčků pro jazyk JavaScript \cite{npm}. V~této práci je využíván zejména na kompilaci jazyka SASS do CSS a automatickou obnovu lokálního serveru.



%%%%%%%%%%%%%%%%%%%%%%%%%% Návrh %%%%%%%%%%%%%%%%%%%%%%%%%%%%%%%
\chapter{Návrh aplikace}
\label{chap:navrh}
U~webových aplikací obecně platí, že rozhodnutí uživatele o~využití či opuštění stránky z~velké části závisí na designu, použitelnosti ale hlavně prvním dojmu. Proto je třeba, aby kromě kvalitní funkcionality měla aplikace také moderní a promyšlený vzhled. \\
Nejprve je třeba vytyčit cílovou skupinu. Díky tomu zjistíme, na jaké části systému je třeba se zaměřit a případně jakým způsobem je nejlépe zpracovat pro co největší uživatelský komfort. Dalším krokem je analýza případů užití, pomocí které zjistíme jaké funkce budou uživatelé potřebovat a používat. Následuje návrh databáze, vycházející z~dat, se kterými chceme pracovat a z~jejich vztahů. Dalším bodem je návrh grafické prezentace. Po ní už následují konkrétní teoretická řešení problémů implementace - výběr vhodných otázek, zpracování výsledků a postup při výběru severu a domény.

Kapitola vrcholí samotným grafickým návrhem, zpracovávajícím všechny předchozí kroky do jednoho úhledného a intuitivního uživatelského rozhraní.


\section{Cílová skupina}
Cílová skupina kalkulačky budou obecně všichni lidé s~volebním právem, tedy občané České republiky starší 18 let \cite{ustava-volebni_pravo}. Podle statistiky zveřejněné na webu statistikaamy.cz\footnote{statistikaamy.cz: \url{www.statistikaamy.cz}}, který spravuje Český statistický úřad\footnote{Český statistický úřad: \url{https://www.czso.cz/csu/czso/domov}}, používalo v~roce 2016 internet 32,5\% z~lidí starších 65 let \cite{statistikaamy}. Bude tedy třeba zvýšit důraz na jednoduchost, přehlednost a intuitivnost, které jsou zejména pro starší lidi kritické.
\par Dále lze tuto skupinu rozdělit na dvě menší části. Tou první jsou lidé, kteří využijí naplno všech funkcí webu, včetně registrace a uložení výsledků na svůj profil, hodnocení poslanců a komentářů. Zde budou spadat zejména mladší uživatelé. Pro ty bude důležité, aby registrace byla jednoduchá a rychlá a zároveň jim web umožňoval spravovat svůj profil a své uložené výsledky. 
\par Zbylá část bude tvořena lidmi, které bude zajímat pouze jejich výsledek a hned po jeho zjištění web opustí. V~tomto případě je nutné co nejvíce zjednodušit přístup k~hlavní funkční části webu.


\section{Případy užití}
Po příchodu na web uvidí každý uživatel domovskou stránku, poskytující základní přehled o~smyslu a cíli aplikace. Na výběr zde má celkem tři akční oblasti. Kromě horního menu a~patičky, které jsou přístupny na každé stránce aplikace, také velké akční tlačítko sloužící pro přechod do sekce s~popisem kalkulačky. Každý uživatel bez rozdílu může hlasovat u~jednotlivých hlasování a používat kalkulačku. Ovšem pouze registrovaným uživatelům jsou výsledky průběžně ukládány a neztratí se například po vymazání mezipaměti prohlížeče.
\par Horní menu odkazuje na jednotlivé části aplikace, na přehledy poslanců, hlasování a~stran, které může použít každý uživatel. Je zde také tlačítko pro přihlášení, případně odkaz na uživatelský účet, pokud již uživatel přihlášený je. Pokud není, je mu přes něj umožněno se buď přihlásit, nebo přejít dalším odkazem na registrační formulář. Z~přehledu poslanců se dá následně přejít na detail jednotlivých poslanců. Zde může registrovaný uživatel ohodnotit konkrétního poslance pomocí hvězdiček, či přidat komentář. Tyto dvě funkce byly omezeny pouze pro přihlášené uživatele z~důvodu možného falšování hodnocení, či urážlivých komentářů. 
\par Přihlášeným uživatelům se přechodem do jejich účtu umožní upravovat svůj profil - přidat či upravit profilovou fotku, změnit heslo, změnit svůj email.
\par V~patičce pak má každý uživatel přístup k~informacím o~webu a ke stránce s~kontakty.

\par
Případy užití znázorňuje diagram případů užití na obrázku \ref{fig:diagram_uc}

\begin{figure}
    \centering
    \includegraphics[width=1\textwidth]{obrazky-figures/UC.pdf}
    \caption{Diagram případů užití pro volební kalkulačku}
    \label{fig:diagram_uc}
\end{figure}



\section{Návrh databáze}
Největší část databáze tvoří tabulky převzaté z~výchozích dat. Zde prakticky nedošlo k~žádným úpravám vůči jejich popisu na webu Poslanecké sněmovny\footnote{Data PSP: \url{https://www.psp.cz/sqw/hp.sqw?k=1301}}. Pro uložení ohodnocení jednotlivých podstatných hlasování bylo nutno přidat tabulku \texttt{hl\_hlasovani\_rating}, uchovávající ohodnocení ke každému hlasování. Django potřebuje také tabulky k~uchovávání informací o~registrovaných uživatelích a jejich oprávněních, či databázových migracích, což jsou zjednodušeně řečeno záznamy o~změnách databázové struktury. Dále byly přidány tabulky pro ukládání komentářů a hvězdičkových hodnocení jednotlivých poslanců. Všechny tabulky a jejich vzájemné vztahy jsou znázorněny v~diagramu s~názvem \texttt{model-diagram.svg} ve složce \texttt{Diagramy} na paměťovém médiu.

\par Tabulek je příliš velké množství na popis, nicméně za zmínku stojí hlavně tabulky 

\begin{itemize}
    \item \texttt{hl\_hlasovani} \\
    uchovávající informace o~jednotlivých hlasováních
    \item \texttt{hl\_poslanec}\\ 
    uchovávající všechna data o~hlasování jednotlivých poslanců pro každé z~existujících hlasování v~předchozí tabulce
    \item \texttt{hist}\\ 
    která je využita pro získání informací pro zjištění důležitých hlasování
    \item \texttt{organy} \\
    uchovávající informace o~orgánech, což můžou být strany, skupiny poslanců, volební období, výbory, a podobně
    \item \texttt{zarazeni}\\ 
    kde nalezneme  informace o~zařazení jednotlivých poslanců v~orgánech
    \item \texttt{osoby}\\ 
    která uchovává informace o~jednotlivých osobách a na kterou se odkazuje tabulka \texttt{hl\_poslanec}
\end{itemize}

\section{Drátový model}
\label{section:wireframe}
Drátový model, neboli takzvaně \texttt{wireframe}, slouží ke zjednodušené vizualizaci umístění jednotlivých prvků webu v~prostoru. Měl by obsahovat všechny elementy, které se budou na výsledném webu nacházet, nejlépe již i v~konkrétních velikostech, aby šlo už v~této fázi zjistit a odstranit případné nedostatky v~rozložení. Smyslem tohoto modelu není zobrazit jak bude web vypadat, pouze jeho strukturu. Proto se používají spíše černobílé, či jednobarevné modely, ve kterých jsou designové bloky a obrázky nahrazeny jednoduchými obdélníky.
\par V~ideálním případě se pro každý typ podstránky dělají až 4 drátové modely, v~závislosti na různých typech responzivního zobrazení. V~této práci postačuje jedna velikost, vytvořená v~programu AdobeXD\footnote{AdobeXD: \url{https://www.adobe.com/products/xd.html}}. Vytvořený drátěný model úvodní strany lze vidět na obrázku \ref{fig:wireframe-homepage}. Kompletní drátové modely jsou přiloženy na datovém médiu ve složce \texttt{Grafické návrhy}. 

\begin{figure}
    \centering
    \includegraphics[width=0.8\textwidth]{obrazky-figures/wireframe-homepage.pdf}
    \caption{Drátěný model úvodní strany aplikace}
    \label{fig:wireframe-homepage}
\end{figure}



\section{Uživatelské rozhraní}
Při návrhu uživatelského rozhraní je obecně třeba dodržovat základy UX\footnote{UX: user experience} (\texttt{uživatelská zkušenost}) a UI\footnote{UI: user interface} (\texttt{uživatelské rozhraní}) designu. Tyto dva pojmy se v~stále častěji zmiňují při tvorbě webových stránek a aplikací, nicméně nezřídka dochází k~jejich zaměňování. 
\par UX design řeší, jakým způsobem bude uživatel s~produktem pracovat. Tím obecně nemusí být myšlena jen digitální aplikace, ale může se jednat také o~nějakou fyzickou věc, či službu. O~cokoliv co může uživatel nějakým způsobem zažít. Do UX designu patří například drátový model, viz sekce \ref{section:wireframe}.
\par Pokud jde o~UI, tak na rozdíl od UX se jedná již o~čistě digitální pojem. Uživatelské rozhraní je, zjednodušeně řečeno, místo, na kterém uživatel může komunikovat s~aplikací. UI zahrnuje celkový vzhled, interaktivitu a intuitivnost aplikace a také celkový uživatelský pocit z~používání. V~současné době se zde řadí také responsivita.

\subsection{Postup při návrhu uživatelského rozhraní}
Jako první bylo třeba vypracovat základní strukturu webu se všemi prvky. K~tomu posloužil již zmiňovaný drátový model. Tento byl po vytvoření prezentován nezainteresovaným lidem v~roli uživatelů, aby byla získána zpětná vazba. Na jejím základě došlo k~několika úpravám.
\par Jako další krok bylo potřeba vymyslet název a z~něj vycházející logo celé aplikace. Jelikož smyslem práce je poskytnout lidem radu, čili učinit je chytřejšími (chytrý je anglicky \texttt{smart}), a následně jim díky tomu ulehčit výběr při volbě (anglicky \texttt{vote}), nabízelo se spojení právě těchto dvou slov. Původní volba byla SmartVote, nicméně po kontrole dostupnosti domény byla zvolena prohozená verze, čili VoteSmart - česky \texttt{Volte chytře}. Samotné logo bylo následně vytvořeno ve webové aplikaci freelogodesign.org\footnote{freelogodesign.org: \url{https://www.freelogodesign.org/}}. Je tvořeno jménem webu a symbolem zaškrtávacího políčka, symbolizujícím hlavní účel webu - volbu.
\par Následuje výběr vhodného barevného schématu. Jelikož se jedná o~politicky orientovanou aplikaci pro prostředí České republiky, nabízí se použití národních barev - červené, modré a bílé. Konkrétní barevné hladiny (\texttt{\#11457e} pro modrou a \texttt{\#d7141a} pro červenou) byly převzaty z~jedné z~použitých grafik, konkrétně z~mapy České republiky \cite{grafika-mapa}, viditelné na grafickém návrhu na úvodní straně. 
\par Jako font byl zvolen Segoe UI. Primární používané velikosti jsou 35px pro nadpisy H1, 25px pro H2 a 16px pro běžný text.
\par Samotný grafický návrh, umístěn na paměťovém médiu ve složce \texttt{Grafické návrhy}, byl vypracováván opět v~programu AdobeXD. Na obrázku \ref{fig:graphic-homepage} můžete vidět design úvodní strany. Při vytváření grafického návrhu byly použity principy responsivního designu, nicméně samotný návrh je pouze pro desktopovou\footnote{Desktopová verze je slangový výraz pro běžné obrazovky stolních počítačů, obecně s~poměrem stran 16:9} verzi. 

\par Při vytváření grafické předlohy byly zjištěny určité nedostatky, na které jsem při vytváření drátového modelu nenarazil. Je tedy v~některých částech odlišná. Také došlo k~zásadním změnám v~principu fungování aplikace oproti původnímu plánu, které v~drátovém modelu nejsou zpětně reflektovány. 


\begin{figure}
    \centering
    \includegraphics[width=0.8\textwidth]{obrazky-figures/Homepage.pdf}
    \caption{Grafický návrh úvodní strany aplikace}
    \label{fig:graphic-homepage}
\end{figure}


\section{Výběr otázek}
Jedná se o~nejobtížnější a nejspíše také nejdůležitější část celé práce. Na výběru vhodných otázek totiž záleží nejenom to, nakolik dokážeme uživateli přiřadit podobnost s~jednotlivými stranami a poslanci, ale také zda uživatel vůbec bude chtít aplikaci používat.\\

\par Jelikož uživatel si bude vybírat hlasování, u~kterých chce sám hlasovat, je zapotřebí mu jich předložit co největší počet. Nicméně jen za aktuální volební období je těchto hlasování takřka 8000 a naprostá většina z~nich nemá ve výsledku žádný vliv na to, zda bude konkrétní návrh zákona přijat či zamítnut. Proto je zde nutné zavést určitou algoritmizaci, usnadňující uživateli práci s~aplikací. 

\par Důkladným průzkumem dat, která jsou podkladem této práce, jsem došel k~závěru, že na jejich základě není možné u~jednotlivých hlasování strojově určit, zda konkrétní hlasování bylo rozhodující například pro přijetí návrhu a jeho posunutí do dalšího kroku, či zda šlo kupříkladu pouze o~hlasování o~výboru, který dostane již přijatý návrh ke zpracování. Ohledem tohoto zjištění jsem kontaktoval přímo správce dat Poslanecké sněmovny, který tento fakt potvrdil. Také však uvedl, že pro některá hlasování je možné tyto informace dohledat prostřednictví sněmovních tisků. Ty jsou součástí každého návrhu zákona a každá změna jejich stavu je zaznamenávána v~tabulce \texttt{hist}.\\

\par Vzhledem k~nemožnosti vyřadit irelevantní hlasování byla tedy využita pouze ta, u~kterých lze pomocí tabulky \texttt{hist} určit jejich důležitost. Tím se snížil počet prezentovaný uživateli na přibližně 900 za poslední dvě volební období.\\ 

\par Jelikož se stále jedná o~příliš velký počet, je třeba uživateli dále usnadnit práci. Konkrétně lze například seřadit hlasování podle relevance dle algoritmického vyhodnocení jejich skóre na základě rozdílnosti hlasování jednotlivých stran. Čím větší různorodost ve hlasování napříč stranami, tím lépe z~hlediska určení podobnosti uživatele se stranami a poslanci po vyhodnocení v~kalkulačce. \\

\par Jakým způsobem je rozdílnost hlasování podstatná pro výpočet korelace s~jednotlivými stranami je patrné například z~vizualizace na obrázcích \ref{fig:analyza-rozdilne} a \ref{fig:analyza-stejne}. Zatímco z~prvního grafu bychom nedokázali určit prakticky nic, jelikož takřka všichni přítomní hlasovali stejně, z~toho druhého dokážeme již alespoň určitou rozdílnost vyčíst. Jen pro pořádek je však třeba zmínit, že podobnost v~jednom hlasování není dostatečná a pro dostatečně validní výstupy bude třeba, aby uživatel hlasoval u~několika. Čím více hlasování, tím přesnější výsledky.

\begin{figure}
    \centering
    \includegraphics[width=1\textwidth]{obrazky-figures/analyza_hl_rozdilne.png}
    \caption{Vizualizace \uv{přínosného} hlasování pro kalkulačku}
    \label{fig:analyza-rozdilne}
\end{figure}

\begin{figure}
    \centering
    \includegraphics[width=1\textwidth]{obrazky-figures/analyza_hl_stejne.png}
    \caption{Vizualizace \uv{nepodstatného} hlasování pro kalkulačku}
    \label{fig:analyza-stejne}
\end{figure}

\section{Způsob zpracování výsledků kalkulačky}
Uživatel bude \uv{hlasovat} ve stejných hlasováních jako skuteční poslanci. Porovnávání proto bude probíhat komparací jeho voleb s~hlasováním jednotlivých poslanců a stran u~těch hlasování, kde uživatel uvedl svoji volbu. Tzn. pokud bude uživatel hlasovat například u~\texttt{\#67351} a~\texttt{\#67444}, budou jeho výsledky porovnávány s~výsledky poslanců pouze v~těchto dvou konkrétních hlasováních. Následně se pro každého poslance vypočte procentuální shoda a~výsledky se seřadí od té největší.

\par Porovnání se stranami záleží na poměru stejných odpovědí členů strany s~celkovým počtem členů pro každou uživatelem odhlasovanou otázku zvlášť.



\section{Výběr vhodného serveru}
Aby aplikace mohla být lidmi využívána, musí být umístěna na veřejně přístupném serveru. Zde máme pro web obecně 3 možnosti:
\begin{itemize}
    \item Sdílený hosting\footnote{Hosting je zažitý název pro server na kterém je aplikace, či webová stránka umístěna}\\
    Jedná se o~nejlevnější a obecně nejjednodušší možnou variantu. Daň za nízkou cenu je však ve sdílení hardwaru s~dalšími zákazníky. Z~toho důvodu není možné dopředu předvídat dostupné prostředky a náročnější aplikace můžou skončit chybou kvůli jejich spotřebování ostatními. Také zde uživatel ve většině případů nemá kontrolu nad nainstalovanými aplikacemi a může využívat pouze omezený počet těch předpřipravených. Je zde také bezpečnostní riziko, protože data různých aplikací jsou na stejných úložištích. Jako příklad můžeme uvést například endora.cz\footnote{endora.cz: \url{https://www.endora.cz/}}
    \item Virtuální privátní server (VPS)\\
    U~tohoto typu hostování je opět hardware serveru sdílen mezi několika aplikacemi. Každý uživatel však již má k~dispozici vlastní virtualizovaný systém, který je oddělen od ostatních. Poskytuje tak uživateli mnohem více prostoru pro správu i lepší zabezpečení.
    Tuto sekci můžeme dále rozdělit na spravované a nespravované virtuální privátní servery. Ty spravované opět uživateli určitým způsobem omezí kontrolu nad nainstalovanými službami, nabídnou však většinou jednoduché možnosti nasazení předem připravených prostředí pro určité specifické zaměření přes přívětivé webové rozhraní, například na cloudways.com\footnote{cloudways.com: \url{https://www.cloudways.com/en/}} lze jedním klikem spustit virtualizovaný server pro WordPressovou aplikaci, kdy po pár minutách uživatel dostane již plně funkční prostředí s~funkční WordPress aplikací včetně databáze, cache a základních pluginů.\par
    U~nespravovaných serverů má uživatel k~dispozici pouze samotný server, přístup k~němu a pár základních funkcí přes webové rozhraní, například možnost ho vypnout či resetovat. Všechnu další správu si uživatel musí zajistit sám - instalaci potřebných aplikací, nainstalování webového serveru, databáze, jejich propojení, nasměrování domény či vytváření subdomén, emaily a další.\\
    Podstatně dražší, než sdílená verze. Např. zmiňované cloudways.com
    \item Dedikovaný sever\\
    Nejdražší varianta. Jde o~kompletně vyčleněný hardware pro použití jedním zákazníkem. Zde má zákazník plnou kontrolu nad obsahem serveru a nainstalovanými aplikacemi. Například sh.cz\footnote{sh.cz: \url{https://www.sh.cz/dedikovane-servery}}
\end{itemize}

Jelikož framework Django nespadá do úplně běžných případů užití, většina sdílených řešení ho nepodporuje. Jako východisko se tedy nabízí buď využít školní server Eva, nebo sáhnout po nějakém dražším sdíleném.
\par Zejména z~důvodu výhod naprosté kontroly nad systémem a jednoduchosti nasazení jsem zvolil digitalocean.com\footnote{digitalocean.com: \url{https://www.digitalocean.com/}}, kde se dostačující sdílený VPS server dá pořídit za zhruba 5\$ měsíčně.

\section{Doména}
Jelikož přístup na web přes IP adresu není moc praktický, potřebuje server zapamatovatelnou doménu, pod kterou ho lidé budou moci vyhledat. Domény s~koncovkou .cz stojí zhruba 200 korun ročně a koupit ji není nic složitého. Například formulář na webu endora.cz\footnote{Endora.cz: \url{https://www.endora.cz/}} je velmi intuitivní. Zde byla také zakoupena doména použitá na vývoj této práce. Po pozdějším průzkumu bych nicméně doporučil pro umístění domén spíše wedos.cz\footnote{wedos.cz: \url{https://www.wedos.cz/}}, vzhledem k~lepším možnostem pro přesměrování domény na data umístěná pod jiným správcem i~k~celkové roční ceně za tzv. \uv{parkování}. Domény lze však většinou kdykoliv bez poplatku přesunout mezi správci.  
\par Aby však web byl pod touto doménou dostupný, je třeba jí nejprve přidat záznam NS s~odkazem na DNS servery\footnote{Dnes server je v~podstatě překladač url adresy na IP adresu, viz \url{https://www.cloudflare.com/learning/dns/what-is-a-dns-server/}} zvoleného poskytovatele hostingu. Pro domény s~koncovkou \texttt{.cz} je tato akce navíc specifická, protože nelze přidat jednotlivé DNS servery, nýbrž je potřeba vytvořit tzv. NSSET\footnote{Vytvoření NSSETu: \url{https://objednavka.forpsi.com/domain/cr-nsset.php}} a ten teprve přidat k~záznamům pro doménu. Tím bude zajištěno, že každý dotaz hledající tuto URL adresu bude nasměrován správným směrem.

\par Aby DNS servery uměly hledanou URL adresu přeložit na IP adresu, je třeba na serveru nastavit záznamy typu A(IPV4), AAAA(IPV6) a NS. Toto nastavení lze například pro digitalocean.com vidět na obrázku \ref{fig:digitalocean-dns}.


\begin{figure}
    \centering
    \includegraphics[width=0.9\textwidth]{obrazky-figures/digitaloceanDNSadmin.png}
    \caption{Rozhraní administrace DNS na webu digitalocean.com}
    \label{fig:digitalocean-dns}
\end{figure}


\chapter{Implementace}
\label{chap:implementace}
Cílem této kapitoly je poskytnout náhled do důležitých kroků implementace. Nejprve je popsána základní struktura aplikace a import dat včetně nalezených problémů. Následuje popis způsobu ohodnocení významných hlasování a určení shod s~poslanci a stranami.

\section{Struktura aplikace}
Jelikož základním programovacím jazykem aplikace je python, je třeba instalovat jeho moduly. Ty lze samozřejmě instalovat i globálně do celého systému, nicméně běžně se pro jejich sdružování k~jednotlivým projektům používá tzv. \texttt{virtuální prostředí}\footnote{Python Virtuální prostředí: \url{https://docs.python.org/3/library/venv.html}}. Tím docílíme oddělení python modulů využitých pro tento projekt od případných jiných verzí jinde v~systému. Díky tomu odpadají problémy s~možnou nekompatibilitou napříč projekty. 

\par Samotná struktura projektu je vygenerována pomocí nástroje \texttt{Cookiecutter-django}. Adresářová struktura vypadá následovně.\\



\begin{minipage}[t]{.5\textwidth}
\dirtree{%
.1 volebni\_kalkulacka.
.2 .envs.
.3 .local.
.3 .production.
.2 compose.
.3 local.
.4 django.
.4 docs.
.4 node.
.3 production.
.4 django.
.4 postgres.
.4 traefik.
.2 config.
.3 settings.
.2 docs.
.2 locale.
}
\end{minipage}
\begin{minipage}[t]{.5\textwidth}
\dirtree{%
.1 .
.2 volebni\_kalkulacka.
.3 contrib.
.4 sites.
.5 migrations.
.3 static.
.4 css.
.4 fonts.
.4 images.
.5 favicons.
.4 js.
.4 sass.
.3 templates.
.4 account.
.4 pages.
.4 users.
.3 users.
.4 migrations.
.4 tests.
.3 utils.
.2 requirements.
}
\end{minipage}

\vspace{0.5cm}
Tato kostra byla poté podle potřeby rozšiřována. Byla přidána samostatná pod-aplikace pro správu importovaných dat \texttt{dataImport} a vlastní moduly pro jednotlivé důležité části aplikace. Následuje popis jednotlivých adresářů. Postupně od vygenerovaných po vlastní.

\subsubsection{Kořenový adresář}
V~kořenovém adresáři se kromě zde vypsaných složek nachází také další pomocné soubory. \texttt{Gulpfile.js} používaný k~nastavení chování \texttt{npm} při zapnutí lokálního vývojového serveru - nastavení automatické kompilace scss do css spolu s~jeho konkatenací a minifikací, automatické obnovy stránek při změně v~\texttt{template} souborech, automatické doplnění nového zkompilovaného css do stránky bez nutnosti jejího nového načtení a jiné.
\par Dále také \texttt{local.yml} a \texttt{production.yml} používané ke spuštění \texttt{docker-compose} a soubor \texttt{manage.py}, který slouží jako výchozí bod pro spouštění všech příkazů týkajících se djanga. Přes něj lze spustit lokální vývojový server, vytvářet databázové migrace, migrovat databáze, používat interaktivní django shell umožňující jednoduše pracovat s~django modely v~konzoli a další. 

\subsubsection{.envs}
Tato složka slouží k~ukládání lokálních a produkčních proměnných. Tyto soubory se běžně nenahrávají na žádný verzovací systém a nastavují se zvlášť na každém zařízení, jelikož obsahují citlivé přihlašovací a jiné údaje.

\subsubsection{compose}
Zde jsou uloženy soubory potřebné pro docker kontejnery. Podsložky django, postgres a~traefik obsahují soubor Dockerfile, který při spuštění aplikace přes \texttt{docker-compose} nastaví příslušný kontejner.

\subsubsection{config}
Tato složka sdružuje hlavní aplikační nastavení. V~souboru \texttt{base.py} se nachází všechna nastavení bez kterých by aplikace nemohla fungovat. Soubory \texttt{local.py} a \texttt{production.py} pak slouží k~upřesnění některých nastavení pro produkční a lokální nasazení. Kromě toho se zde nachází také \texttt{urls.py}, umožňující tzv. \texttt{routing}, což je zjednodušeně řečeno určení co aplikace udělá při obdržení požadavku na určitou url adresu. Obecně se zde nastavují cesty k~jednotlivým modulům aplikace do pole \texttt{urlpatterns}. Například jedna z~položek
\begin{verbatim}
    path("schuze/", include("volebni_kalkulacka.schuze.urls", 
         namespace="schuze")),
\end{verbatim}

znamená, že pokud uživatel přistoupí na adresu \texttt{urlwebu.domena/schuze}, tak se všechny cesty budou hledat v~souboru \texttt{urls.py} v~modulu \texttt{volebni\_kalkulacka.schuze}.

\subsubsection{locale}
Tento adresář je vytvořen aby obsahoval vygenerované jazykové soubory, pokud by aplikace používala více jazyků. Cookiecuter-django v~základu podporuje multijazyčnost. V~této práci, jelikož je specifická jenom pro Českou Republiku, tato možnost nebyla využita, nicméně pokud by byla potřeba, stačilo by pouze nastavit proměnnou \texttt{USE\_I18N} v~\texttt{base.py} na \texttt{True} a začít využívat tzv. \texttt{trans tagy} kolem každého textu zobrazovaného uživateli. Před nasazením aplikace do produkce by se spustil sběrač těchto textů, který by z~nich vytvořil přehledný dokument. Zde by se texty ručně přeložily a následným kompilačním příkazem by se vytvořil binární soubor, ze kterého by django čerpalo při změně jazyka. Celý postup je přehledně popsán v~dokumentaci\footnote{Django překlady \url{https://docs.djangoproject.com/en/3.2/topics/i18n/translation/}}.

\subsubsection{requirements}
Jelikož aplikace už v~základu využívá několik python modulů, nalezneme zde jejich instalační seznam pro python instalátor \texttt{pip}. Je rozdělen stejně jako nastavení do 3 souborů - \texttt{base.txt}, \texttt{local.txt} a \texttt{production.txt}. Díky nim není nutné jednotlivé použité moduly složitě vyhledávat ručně, ale stačí spustit jeden příkaz a všechny moduly jsou nainstalovány.

\subsubsection{volebni\_kalkulacka}
Hlavní aplikační složka. Zde budou umístěny všechny další navazující vytvořené moduly. V~adresáři \texttt{static} se nachází všechny statické soubory, rozdělené podle typu - CSS, fonty, obrázky, JavaScript a SASS, před tím než se zkompiluje na CSS. Tyto soubory jsou po vykreslení webové stránky načítány z~prohlížeče jako zdroje. Přímo z~této složky je to však možné pouze v~lokálním prostředí. Na produkci je nejprve nutné všechny statické soubory umístit na místo k~tomu určené. Django to zvládne jedním jednoduchým příkazem\footnote{Django statické soubory: \url{https://docs.djangoproject.com/en/3.2/ref/contrib/staticfiles/}}.

\subsubsection{volebni\_kalkulacka.templates}
\par V~\texttt{templates} se nachází HTML šablony používané návrhovým vzorem MVT. Jelikož šablony nejsou pouze samotné HTML, ale obsahují také Django Templating Language (django šablonovací jazyk), můžeme při vytváření šablon využívat určitý druh \uv{dědičnosti}, kdy pro základní strukturu stránky, která se bude vždy opakovat vytvoříme \texttt{base.html} soubor. Zde se bude nacházet HTML hlavička, scripty, fonty a jiné statické věci využívané napříč celým projektem. Jednotlivé stránky pak dále budou tuto základní šablonu pouze rozšiřovat. To je umožněno prostřednictví tzv. \texttt{block} (blokových) značek, kdy blok z~rodičovské šablony je nahrazen blokem se stejným názvem ze šablony rozšiřující.

\subsubsection{volebni\_kalkulacka.users}
Předpřipravený modul pro správu uživatelů. Vytváří rozhraní pro registraci, správu uživatelských účtů, změnu hesla, změnu emailu a potvrzení emailu.
 
\subsubsection{dataImport}
První z~vlastních modulů. Slouží pouze k~organizaci kódu potřebného pro stahování a~aktualizaci dat z~webu poslanecké sněmovny. 

\subsubsection{volebni\_kalkulacka.psp\_data}
Tento modul slouží výhradně k~organizaci souborů týkajících se dat z~poslanecké sněmovny a djanga. Všechny modely pro výchozí data jsou definovány zde. Stejně tak je zde definován script pro prvotní stažení všech dat.


\subsection{Import dat}
\label{import_dat}
Aby aplikace mohla fungovat, potřebuje mít přístup k~aktuálním datům z~poslanecké sněmovny. Ta jsou uložena s~příponou \texttt{.unl} a jejich formát je podobný souborům \texttt{.csv} (\texttt{Comma Separated Values}), česky zvaným \texttt{čárkou oddělené hodnoty}. S~tím rozdílem, že na rozdíl od CSV obsahují oddělovač i za posledního hodnotou na řádku, což standardu CSV neodpovídá \cite{RFC4180}. Například tabulka \texttt{hl\_poslanec}, uchovávající záznamy o~hlasování jednotlivých poslanců pro každé hlasování má následující formát dat\\
\\
\texttt{1539|67018|A|\\
1537|67018|A|\\
1540|67018|A|}\\
\\
kde na prvním sloupci je odkaz na unikátní identifikátor poslance v~tabulce \texttt{poslanec}, ve druhém sloupci je odkaz na unikátní identifikátor hlasování v~tabulce \texttt{hl\_hlasovani} a~ve třetím sloupci je volba, kterou poslanec učinil. 

\par Kromě toho jsem v~průběhu vývoje narazil u~datových souborů na několik dalších problémů.

\begin{itemize}
    \item neexistující cizí klíče u~několika tabulek\\
    např. referencovaný nadřazený typ orgánu s~ID 111 v~tabulce \texttt{typ\_organu}
    \item nekonzistence dat\\
    V~tabulce \texttt{zarazeni} může mít sloupeček \texttt{id\_of} hodnotu cizích klíčů buď z~tabulky \texttt{organy}, nebo z~tabulky \texttt{funkce} v~závislosti na hodnotě sloupce \texttt{cl\_funkce}.
    \item data nekorespondující k~deklarovanému typu podle popisu tabulek\\
    Ve sloupcích podle popisu definovaných jako INT, čili celočíselné, se nachází nejenom čísla ale někdy také písmena či kombinace čísel a písmen. Například v~tabulce \texttt{tisky} se ve sloupci \texttt{zm\_lhuty} nachází kromě v~popisu deklarovaných hodnot "1"\ a "2"\ také hodnoty "a"\ a "n".
    \item Kódování souborů nekorespondující s~deklarovaným v~popisu, ani s~žádným z~několika možných vyzkoušených.\\
    Například u~tabulky \texttt{tisky} by stejně jako u~ostatních tabulek mělo fungovat deklarované kódování \texttt{windows-1250}, nicméně při pokusu o~otevření s~tímto kódováním dochází k~chybám. Bylo proto nutné nastavit parametr \texttt{errors='replace'}, na základě kterého dojde k~nahrazení nečitelných znaků. Tyto náhradní znaky však poté byly zobrazeny na webu a podstatně zhoršovaly uživatelskou zkušenost, proto bylo třeba je nahradit zpětně zase na odpovídající české znaky.
    \item Nejednoznačnost prázdných hodnot\\
    V~některých tabulkách je prázdné pole zaznačeno oddělovači sloupců u~sebe, což odpovídá hodnotě NULL, čili žádné. Nicméně jinde jsou jako prázdné hodnoty použity mezery, či lomítka. Například v~případě tabulky \texttt{osoby} a sloupce \texttt{pred} je to dokonce kombinace - buď NULL, nebo mezera.
    \item Nekonzistence názvů sloupců odkazujících na stejný prvek napříč tabulkami\\
    V~některých tabulkách je cizí klíč odkazující na volební období pojmenován jako \texttt{id\_org}, v~jiných zase jako \texttt{id\_organ}, což velmi znepříjemňuje práci s~daty kvůli neustálé nutnosti kontrolovat názvy sloupců.
    
    
\end{itemize}

\subsubsection{Aktualizace databáze}
Aktualizaci databáze lze provést přes příkaz
\begin{verbatim}
    python3 manage.py initial_import
\end{verbatim}
z~kořenového adresáře aplikace. Jeho definice se nachází v~souboru\\
\texttt{volebni\_kalkulacka.psp\_data.management.commands.initial\_import.py}. \\

\par V~prvotní verzi tento script\footnote{Script: krátký kus samostatně spustitelného kódu} zpracovával každý nahrávaný soubor řádek po řádku a~každý zvlášť vkládal po dávkách do databáze. Tímto způsobem bylo možné lépe kontrolovat jak budou data zpracována a případně opravit nalezené chyby. Nicméně po změně koncepce aplikace a nutnosti zpracovávat všechna data bez omezení na pouze poslední volební období jsem narazil na výkonnostní problémy, kdy by zpracování zhruba 17 milionů řádků databáze přestalo být realizovatelné.
Proto bylo nutné navrhnout lepší způsob.

\par Současná verze již utilizuje PostgreSQL příkaz \texttt{COPY}, díky kterému je možné zpracovávat celé soubory najednou \cite{psql-copy}, což velmi zásadně zvýšilo rychlost, jakou jsou data do databáze nahrána. Bohužel v~důsledku této změny již není možné průběžně data upravovat. Navíc je třeba převést unl formát na csv.

\par Samotný script tedy funguje následovně. Nejprve jsou dynamicky získány odkazy všech stažitelných zip souborů na stránce s~daty. Tímto zajistíme automatické přidání nových volebních období bez nutnosti úpravy kódu. Následně jsou staženy a rozbaleny. \\

\par Některé soubory zůstávají neměnné napříč všemi volebními obdobími. Jejich názvy jsou uloženy v~souboru s~konstantami v~konstantě s~názvem \texttt{INITIAL\_FILE\_NAMES} s~datovým typem \texttt{dict} neboli česky slovník, kde je navázán název databázové tabulky na odpovídající jméno staženého souboru. Přes tyto dvojice (tabulka:soubor) je iterováno hned 2x. Poprvé je to kvůli již zmiňovanému unl formátu, konkrétně poslednímu oddělovači. Tento musí být odstraněn, jelikož příkaz COPY očekává, že za každým oddělovačem v~souboru se nachází nový sloupec. Nahrávaný soubor je tedy otevřen a do nového souboru s~příponou \texttt{\_formatted} je uložen jeho obsah již bez tohoto nezbedného znaku.
\par Druhá iterace nejprve provede nad danou tabulkou příkaz \texttt{TRUNCATE RESTART IDENTITY}, který z~ní smaže všechna data a vynuluje čítač řádků \cite{psql-truncate}. Dále získá hlavičky souboru, respektive tabulky, z~konstanty \texttt{TABLE\_HEADERS}, což je opět slovník dvojic, nyní ve tvaru (tabulka:pole názvů hlaviček). Následuje vypnutí databázových spouštěčů (anglicky trigger), aby bylo možné nahrát i data s~nekonzistentními hodnotami, samotné nahrání dat a~opětovné zapnutí spouštěčů. 

\par Pro soubory s~hlasováním je postup stejný, s~tím rozdílem, že se budou objevovat nové názvy souborů pro nová volební období. Názvy jsou proto v~této fázi vybírány dynamicky ze stažených souborů a všechny se nahrávají do jedné tabulky. Stejný princip je uplatněn jak pro tabulky \texttt{hl\_poslanec} tak \texttt{hl\_hlasovani}.

\par Po uložení všech potřebných dat jsou odstraněna zmatečná hlasování.\\ 

\par Dalším krokem je výpočet ohodnocení jednotlivých důležitých hlasování - těch která mají zásadní vliv na další existenci návrhů zákona. Tato funkce je jedna z~nejpodstatnějších částí aplikace, jelikož je důležité uživateli hned zpočátku předložit v~kalkulačce ty nejzajímavější možnosti. Samotný kód realizující tuto operaci se nachází v~metodě \texttt{calculateHlasovaniRatings} třídy \texttt{DbManager} zajišťující práci s~databází pro import dat, umístěné v~souboru \texttt{volebni\_kalkulacka/dataImport/Logic/DbManager.py}. 
\par Nejdříve potřebujeme získat všechna důležitá hlasování za poslední dvě volební období. K~tomuto účelu využívám SQL dotaz propojující tabulku \texttt{hl\_hlasovani} s~tabulkou \texttt{hist}, kde hledáme pouze hlasování spadající do ID orgánů značících poslední 2 volební období:

\begin{lstlisting}[language=SQL, caption={SQL dotaz na všechna podstatná hlasování za poslední 2 volební období}, label=code:sql-dve-obdobi]
SELECT * 
FROM 
    psp_data_hl_hlasovani as hh

    INNER JOIN psp_data_hist as h
    ON h.id_hlas = hh.id_hlasovani

WHERE 
    hh.id_organ IN (
        --last 2 election periods
        select 
            id_organ from psp_data_organy 
        where 
            organ_id_organ is null
            and id_typ_organu = 11 --poslanecka snemovna
        union
        select id_organ-1 from psp_data_organy 
        where 
            organ_id_organ is null
            and id_typ_organu = 11 --poslanecka snemovna
    )
\end{lstlisting}

Následně již v~pythonu provedeme nad získanými daty cyklus, kdy pro každé z~hlasování pokládáme další dotaz do databáze. Tentokrát hledáme dvojici (zkratka, vysledek), kde zkratka je zkratka strany a vysledek je hlasováni poslance. Jedna dvojice za každého poslance zúčastněného v~konkrétním hlasování.\\
\begin{lstlisting}[language=SQL, caption={SQL dotaz na získání dvojic zkratka, výsledek pro zadané hlasování}, label=code:sql-zkratka-vysledek]
SELECT 
    zkratka,vysledek
FROM 
    psp_data_hl_poslanec AS hp 

    INNER JOIN psp_data_poslanec AS p 
    ON hp.id_poslanec = p.id_poslanec 
    
    INNER JOIN psp_data_zarazeni AS
z~ON p.id_osoba = z.id_osoba
    
    INNER JOIN psp_data_organy as
o~ON z.id_of = o.id_organ
    
    INNER JOIN psp_data_osoby as os
    ON os.id_osoba = p.id_osoba
WHERE 
    hp.id_hlasovani = {hlasovani['id_hlasovani']}
    AND z.cl_funkce = 0 --clenstvi
    AND o.organ_id_organ = {hlasovani['id_organ']} --ID election period 
    AND o.id_typ_organu = 1 --Klub
    AND ( --either membership is active (null) 
        --or membership ended together with election period
        TO_DATE(z.do_o, 'YYYY-MM-DD') = TO_DATE(o.do_organ,'DD.MM.YYYY')
        OR 
        z.do_o IS null
    )
\end{lstlisting}

Výsledek databázového dotazu je načten do knihovny Pandas, což je rychlý, výkonný, flexibilní a jednoduše ovladatelný open-source nástroj pro manipulaci s~daty a jejich analýzu \cite{pandas}. Zde je opět cyklus, v~tomto případě pro každou ze zúčastněných stran v~hlasování. Výpočet ohodnocení hlasování se dá popsat sumou
\begin{equation}
    OH = \sum_{i=1}^N \Big(1-\Big|\frac{pocet\_pro_i}{pocet\_clenu_i}-\frac{pocet\_proti_i}{pocet\_clenu_i}\Big|\Big) * \frac{pocet\_clenu_i}{pocet\_vsech\_zucastnenych} * 100
\end{equation}

kde \texttt{OH} je ohodnocení hlasování, \texttt{N} je celkový počet stran, \texttt{i} odpovídá jedné ze stran přes kterou cyklus zrovna prochází, \texttt{pocet\_pro} je počet poslanců ve straně hlasujících pro přijetí, \texttt{pocet\_proti} je počet poslanců ve straně hlasujících pro zamítnutí, \texttt{pocet\_clenu} je počet přítomných členů strany hlasujících pro přijetí/zamítnutí a~\texttt{pocet\_vsech\_zucastnenych} je celkový počet členů všech stran přítomných na tomto hlasování.

\par Cílem tohoto hodnocení je nalézt ta hlasování, kde poslanci jednotlivých stran hlasovali nejvíce rozdílně. Čili kde je nejmenší rozdíl mezi hlasy pro a proti uvnitř jednotlivých stran. Tento rozdíl dále násobíme poměrem velikosti strany k~celkovému počtu členů parlamentu, aby se přizpůsobila váha jednotlivých stran k~jejich velikosti. Výsledek hodnocení poté uložíme spolu s~číslem hlasování do tabulky \texttt{hl\_hlasovani\_rating}. Při výpisu uživateli poté řadíme hlasování od těch s~největším ohodnocením.\\

\par Nakonec je spuštěna funkce na nahrazení nevalidních znaků popsaných v~problémech s~daty na začátku kapitoly \ref{import_dat}.\\

Průběžná aktualizace dat bude prováděna pomocí \texttt{cronu}, což je standardní Unix nástroj používaný na naplánování automatického spuštění příkazů v~zadaný čas \cite{cron}. Na serveru kde bude aplikace umístěna bude do \texttt{crontab} souboru třeba přidat záznam
\begin{verbatim}
    0 3 * * * docker-compose -f /opt/volebni_kalkulacka/production.yml run 
    --rm django python manage.py initial_import
\end{verbatim}
kde \texttt{/opt/volebni\_kalkulacka/production.yml} bude odpovídat skutečnému umístění na serveru. V~příkazu figuruje \texttt{docker-compose} příkaz, jelikož na produkci je aplikace umístěna v~docker kontejnerech a je třeba tuto aktualizaci spustit uvnitř jednoho z~nich. 

\subsection{Kalkulačka shod}
U~každého hlasování může uživatel sám hlasovat. K~tomuto účelu slouží malý modul po straně, kde si může vybrat ze tří možností. Pro, proti a zdržet se. Může také svoji volbu zrušit. Vše je patrné z~obrázku \ref{fig:aplikace-hlasovani}.

\begin{figure}
    \centering
    \includegraphics[width=1\textwidth]{obrazky-figures/aplikace-hlasovani.png}
    \caption{Rozhraní pro uživatelské hlasování}
    \label{fig:aplikace-hlasovani}
\end{figure}

\par Své výsledky si uživatel může zobrazit v~sekci Kalkulačka viz obrázek \ref{fig:aplikace-kalkulacka}. Jelikož může hlasovat napříč několika volebními obdobími, byly výsledky rozděleny podle nich. Má tedy jeden výsledek pro každé období zvlášť. 

\begin{figure}
    \centering
    \includegraphics[width=1\textwidth]{obrazky-figures/aplikace-kalkulacka.png}
    \caption{Zobrazení výsledků kalkulačky v~uživatelském profilu}
    \label{fig:aplikace-kalkulacka}
\end{figure}

\par Vyhodnocení shod probíhá v~cyklu pro všechna období. Pro shodu s~poslanci je použit SQL dotaz, který nejprve vyhledá všechny dvojice \texttt{matches}, \texttt{id\_poslanec}, kde id\_poslanec je unikátní identifikátor poslance a sloupeček matches uchovává počet jeho shodných hlasování s~uživatelem. Tyto informace jsou poté ve druhém databázovém dotazu spojeny do jedné tabulky spolu s~daty o~jednotlivých poslancích a jejich zařazení ve stranách v~zadaném volebním období. Výsledkem je pole trojic \texttt{match\_ratio}, \texttt{id\_poslanec}, \texttt{zkratka}, kde match\_ratio je poměr shod poslance s~uživatelem a zkratka je zkratka jeho strany. Ten je poté zobrazen v~kalkulačce. Výpočet shody uživatele s~poslancem lze popsat vzorcem

\begin{equation}
    SP_{ij} = \frac{pocet\_hlasovani_i}{pocet\_shodnych\_hlasovani_{ij}}*100
\end{equation}

kde \texttt{SP} je procentuální ohodnocení shody poslance \texttt{j} a uživatele za volební období \texttt{i}, \texttt{pocet\_hlasovani} je počet hlasování které uživatel uskutečnil spadajících do volebního období \texttt{i} a \texttt{pocet\_shodnych\_hlasovani} je počet hlasování, kde se uživatel a poslanec \texttt{j} ve volebním období \texttt{i} shodli.



\begin{lstlisting}[language=SQL, caption={SQL dotaz na vyhledání shod uživatele s~poslanci}, label=code:sql-shoda-poslanci]

DROP TABLE IF EXISTS t1;
CREATE TEMP TABLE t1 AS   
    SELECT count(*) AS matches, id_poslanec 
    FROM psp_data_hl_poslanec
    WHERE (id_hlasovani, REPLACE(REPLACE(vysledek, 'B', 'N'),'C', 'K')) 
        IN %s
    GROUP BY id_poslanec 
    ORDER BY matches DESC;

DROP TABLE IF EXISTS t2;
CREATE TEMP TABLE t2 AS 
    SELECT DISTINCT ON (p.id_poslanec) 
        (cast(t1.matches as decimal(7,2))/%s)*100 
        AS match_ratio, t1.id_poslanec, o.zkratka, 
        os.pred, os.jmeno, os.prijmeni, os.za
    FROM t1
        INNER JOIN psp_data_poslanec AS p
        ON p.id_poslanec = t1.id_poslanec

        INNER JOIN psp_data_osoby AS os
        ON p.id_osoba = os.id_osoba

        INNER JOIN psp_data_zarazeni AS z 
        ON p.id_osoba = z.id_osoba

        INNER JOIN psp_data_organy as o
        ON z.id_of = o.id_organ
    WHERE z.cl_funkce = 0
        AND o.organ_id_organ = {period_id}
        AND o.id_typ_organu = 1
    ORDER BY 
        p.id_poslanec, TO_DATE(z.od_o, 'YYYY-MM-DD') ASC
;

SELECT * 
FROM t2
ORDER BY match_ratio DESC, prijmeni
\end{lstlisting}

\par Shody se stranami jsou počítány obdobně. Nejprve jsou získány počty poslanců ve všech stranách v~procházeném volebním období. Dále počet hlasů shodných s~uživatelem pro každou stranu. V~posledním kroku se tyto dvě informace spojí a na výstupu je opět pole, tentokráte dvojic \texttt{match\_ratio}, \texttt{zkratka}, kde match\_ratio je procentuálně vyjádřena průměrná shoda uživatelských voleb s~jedním poslance ve straně. Výpočet lze popsat následujícím vzorcem

\begin{equation}
    SS_{ij} = \frac{\frac{pocet\_shod\_strany_{ij}}{pocet\_odpovedi_i}}{pocet\_clenu\_strany_{ij}} * 100
\end{equation}

kde \texttt{SS} je procentuálně vyjádřená uživatelova průměrná shoda s~členem strany \texttt{j} ve volebním období \texttt{i}, \texttt{pocet\_shod\_strany} je počet shodných odpovědí s~uživatelem za celou stranu \texttt{j} ve volebním období \texttt{i}, \texttt{pocet\_odpovedi} je počet odpovědí uživatele spadajících do volebního období \textbf{i} a \texttt{pocet\_clenu\_strany} je počet členů strany \texttt{j} ve volebním období \texttt{i}.

\begin{lstlisting}[language=SQL, caption={SQL dotaz na vyhledání shod uživatele se stranami}, label=code:sql-shoda-strany]
--get last party for each member
DROP TABLE IF EXISTS osoby_zarazeni;
CREATE TEMP TABLE osoby_zarazeni AS
    SELECT DISTINCT ON (os.id_osoba) 
        os.id_osoba, o.zkratka, z.od_o, z.do_o, o.od_organ
    FROM 
        psp_data_osoby AS os
        INNER JOIN psp_data_zarazeni AS z 
        ON os.id_osoba = z.id_osoba

        INNER JOIN psp_data_organy AS o
        ON z.id_of = o.id_organ
    WHERE 
        z.cl_funkce = 0
        and o.organ_id_organ = {period_id}
        and o.id_typ_organu = 1
    ORDER BY
        os.id_osoba, TO_DATE(z.od_o, 'YYYY-MM-DD') ASC;

--get members count in party
DROP TABLE IF EXISTS tP1;
CREATE TEMP TABLE tP1 AS
    SELECT 
        COUNT(*) AS members_count, oz.zkratka 
    FROM 
        osoby_zarazeni as oz
	WHERE
        TO_DATE(oz.od_o, 'YYYY-MM-DD') = TO_DATE(oz.od_organ,'DD.MM.YYYY')
    GROUP BY 
        oz.zkratka;

DROP TABLE IF EXISTS tP2;
CREATE TEMP TABLE tP2 AS
    SELECT 
        count(*) AS party_total_matches, oz.zkratka 
    FROM 
        psp_data_hl_poslanec as hlp
        INNER JOIN psp_data_poslanec AS p
        ON p.id_poslanec = hlp.id_poslanec

        INNER JOIN osoby_zarazeni AS oz
        ON p.id_osoba = oz.id_osoba

    --need to replace characters 
    --because they use multiple with same meaning
    WHERE 
        (id_hlasovani, REPLACE(REPLACE(vysledek, 'B', 'N'),'C', 'K'))IN %s
    GROUP 
        BY oz.zkratka;

SELECT 
    ((CAST(tP2.party_total_matches as decimal(7,2))/{answers_count})
    /tP1.members_count)*100 AS match_ratio, tP1.zkratka 
FROM tP2
    INNER JOIN tP1
    ON tP1.zkratka = tP2.zkratka
ORDER 
    BY match_ratio DESC, tP1.zkratka
\end{lstlisting}


\subsection{Monitoring komentářů}
Umožnění uživatelům přidávat obsah s~sebou vždy nese rizika. V~případě citlivých témat, jako je politika, jsou tato rizika násobně větší. Bylo proto třeba zavést určitá opatření pro umožnění alespoň částečné správy uživatelského obsahu. 
\par Konkrétně byl přidán jednoduchý detektor vulgarit, kde se na základě seznamu sprostých slov převzatého ze stránky wiktionary.org \cite{wiktionary-vulgarity} vyhodnotí jejich přítomnost v~přidávaném komentáři a je zasláno upozornění do emailových schránek administrátorů. Ti poté můžou rychlým zákrokem komentář smazat a zabránit tak šíření závadného obsahu. Jak vypadá takové upozornění lze vidět na obrázku \ref{fig:aplikace-komentar}.
\par Tento způsob kontroly a správy byl zvolen z~důvodu, že v~případě například pouze blokování závadných slov by si kreativní uživatelé rychle uvědomili která slova algoritmus vyhodnotí jako závadná a našli si jiná, která by byla stejně vulgární, ale nedetekovaná. V~případě povolení přidání všech komentářů a až dodatečné kontroly nemá běžný uživatel vůbec ponětí o~detekci vulgarismů a nemá proto tendenci vymýšlet způsoby jak algoritmy obejít.

\par Tato kontrola může být v~budoucnu dále rozšířena například na detekci odkazů na závadné internetové stránky, či blokování spamu.

\begin{figure}
    \centering
    \includegraphics[width=1\textwidth]{obrazky-figures/aplikace-komentar.png}
    \caption{Email administrátorům s~upozorněním na závadný obsah komentáře}
    \label{fig:aplikace-komentar}
\end{figure}


\chapter{Testování}
\label{chap:testovani}
Aby bylo možné aplikaci testovat, bylo nejprve nutné nasadit ji do produkčního prostředí a zpřístupnit uživatelům. K~tomuto účelu jsem využil vlastní pronajatý VPS na Digitalocean.com. Aby uživatelé na nichž bude testování prováděno nemuseli psát do prohlížeče IP adresu, nasměroval jsem zde svoji doménu tomasmajerech.cz. 

\section{Průběžné testy}
Testování aplikace probíhalo již od počátku. Drátový model byl po dokončení zaslán 6 nezávislým uživatelům ve věkovém rozmezí 16-57 let, kdy každý měl posoudit smysluplnost rozložení dat a celkové struktury. Na základě jejich zpětné vazby byly provedeny drobné změny, například byl přidán obrázek České republiky na úvodní stranu a tlačítko kalkulačky posunuto ze středu více vlevo, či byl do hlasování přidán přehled hlasujících poslanců.

\par Druhé kolo testování proběhlo po dokončení grafického návrhu. Jeho cílem bylo ověřit, zda vizuální stránka aplikace nezaostává za běžnými standardy. Tento test musel být proveden 2x, jelikož v~rané části vývoje došlo k~celkové restrukturalizaci projektu a změně koncepce.
\par Z~výsledků vyplynulo, že na všechny z~testovaných subjektů celkové grafické provedení aplikace působí profesionálně, někdy až oficiálně v~kontextu státní sféry. Na celkovém grafickém stylu tedy nebylo třeba nic upravovat. Nicméně objevily se výhrady k~některým konkrétním částem aplikace, například k~zobrazení jednotlivých hlasování k~bodu pořadu schůze po najetí myší, viz obrázek \ref{fig:wireframe-schuze-hover}, které bylo na základě tohoto hodnocení upraveno do současné podoby.

\begin{figure}
    \centering
    \includegraphics[width=1\textwidth]{obrazky-figures/wireframe-schuze-hover.png}
    \caption{Původní návrh zobrazení hlasování pro bod schůze}
    \label{fig:wireframe-schuze-hover}
\end{figure}


\par Po negativní reakci byly také upraveny šablony posílaných emailů pro registraci, změnu hesla a zapomenuté heslo. Z~původní pouze textové podoby je nyní HTML emailová šablona, která působí na nové uživatele lepším dojmem. Aktuální podobu emailových zpráv, které uživatel obdrží, lze vidět na obrázku \ref{fig:graphic-email}.

\begin{figure}
    \centering
    \includegraphics[width=0.6\textwidth]{obrazky-figures/Email.pdf}
    \caption{Současná podoba emailových šablon}
    \label{fig:graphic-email}
\end{figure}

\section{Testování produkční aplikace}
Jelikož na testování produkční verze aplikace by již nebyl vzorek pouhých 6 uživatelů a~jejich názorů dostatečný, bylo třeba prozkoumat nové možnosti. Tou první je tzv. \texttt{usability testing}, česky testování použitelnosti. Druhou je dotazník vytvořený pomocí Formulářů od Google.

\subsubsection{Testování použitelnosti}
Testování použitelnosti je způsob testování funkcionality webu, aplikace, či jiného digitálního produktu pozorováním skutečných uživatelů při jeho používání a provádění úkolů. Cílem tohoto testování je objevit oblasti, kde může dojít ke zmatení uživatele a odkrýt příležitosti k~vylepšení celkového uživatelského zážitku \cite{hotjar-usability-testing}. 
\par Tato metoda je velmi časově náročná, jelikož vyžaduje plnou účast dvou a více lidí, kdy tester používá aplikaci, plní zadané úkoly a nahlas vyjadřuje svoje myšlenky, zatímco pozorovatel zaznamenává vše důležité.

\par Vzhledem k~nutnosti osobní přítomnosti testovacích subjektů a současným platným omezením pohybu osob byl tento test aplikován pouze na osoby ve společné domácnosti. Konkrétně 3 muže ve věku 20-23 let bez předchozí zkušenosti s~touto aplikací. Test byl prováděn s~každým subjektem zvlášť, nejprve na stolním počítači, poté na mobilním telefonu. V~obou případech byly stejné úkoly. 
\begin{enumerate}
    \item Otevřít webovou stránku
    \item Dát 5 hvězdiček a komentář Tomio Okamurovi
    \item Najít hlasování které proběhlo naposledy
    \item V~tomto hlasování přidat svůj hlas
    \item Najít 5 nejzajímavějších hlasování z~pohledu kalkulačky a hlasovat i u~nich
    \item Prohlédnout si svoje výsledky v~kalkulačce
    \item Změnit heslo k~účtu
    \item Odhlásit se
\end{enumerate}

Celkové výsledky testů dopadly spíše pozitivně. U~žádného ze subjektů se neobjevily problémy s~nalezením konkrétního poslance, komentováním, ani hvězdičkovým ohodnocením poslance. Při hledání posledního hlasování však shodně všichni uživatelé hledali ve špatné kategorii, kdy místo sekce Schůze, kde je poslední hlasování logicky umístěno v~posledním bodě poslední schůze, hledali nejprve v~sekci Hlasování, která však slouží k~jinému účelu. Po otevření sekce Schůze již s~nalezením nebyly žádné problémy, stejně tak s~přidáním vlastního hlasu. Nalezení nejzajímavějších hlasování z~pohledu kalkulačky již bylo pro testovací subjekty jednoduché, jelikož znali sekci Hlasování z~předchozího kroku. Výsledky ke kalkulačce hledaly subjekty nejprve v~modulu pro hlasování. Po neúspěchu však logickou dedukcí odvodili, že se nejspíše výsledky budou nacházet v~profilu, kde je také nalezli. Změna hesla k~účtu a odhlášení nepředstavovaly žádný problém.\\



\subsection{Testování s~dotazníky}
Kombinace dotazníku a odkazu na produkční aplikaci umožňuje rychle a efektivně získat názory velkého množství lidí. Tato metoda není zdaleka tak důkladná jako testování použitelnosti, nicméně umožňuje široký záběr a získání velkého množství informací a názorů.

\par Subjekty v~dotazníkovém testování měly splnit stejné úkoly jako v~testování použitelnosti a následně vyplnit formulář. Otázky ve formuláři byly:
\begin{enumerate}
    \item Použili jste k~návštěvě webu mobilní telefon?
    \item Podařilo se vám zvládnout všechny úkoly?
    \item Pokud ne, u~kterého úkolu jste se zasekl/a?
    \item Pokud jste se zasekl/a, zkuste stručně popsat proč.
    \item Jaké hodnocení byste aplikaci udělili?
    \item Kdybyste mohl/a něco změnit, co by to bylo?
\end{enumerate}

Dotazník byl osobně zaslán několika lidem a umístěn na FIT Discord server, nicméně nutnost postupovat podle úkolů v~něm většinu lidí odradila od jeho vyplnění. \\

\par Celkem dotazník vyplnilo 17 lidí. Z~nich 6 použilo k~návštěvě webu mobilní telefon, viz graf na obrázku \ref{fig:dotaznik-mobil}. 


\begin{figure}[!tbp]
  \centering
  \begin{minipage}[b]{0.45\textwidth}
    \includegraphics[width=\textwidth]{obrazky-figures/dotaznik-mobil.png}
    \caption{Statistika použití mobilního telefonu v~dotazníku.}
    \label{fig:dotaznik-mobil}
  \end{minipage}
  \hfill
  \begin{minipage}[b]{0.45\textwidth}
    \includegraphics[width=\textwidth]{obrazky-figures/dotaznik-vsechno.png}
    \caption{Statistika zvládnutí všech otázek v~dotazníku.}
    \label{fig:dotaznik-vsechno}
  \end{minipage}
\end{figure}

\par Většina, tvořená 10 uživateli, dokázala zvládnout všechny úkoly, viz \ref{fig:dotaznik-vsechno}. Z~uživatelů s~komplikacemi byli 4, kteří označili za blokující úkol č. 7, kde však kvůli chybně skrytému tlačítku nebylo tento úkol možné zpočátku splnit, proto tyto lidi zanedbám. Dále 5 lidí označilo jako bod záseku úkol číslo 2, nicméně dle navazujících textových odpovědí nebyla komplikace v~nemožnosti bod splnit, ale v~nechuti testovacích subjektů se do aplikace registrovat. Dle textových odpovědí by 3 uživatelé rádi zrychlili sekci Hlasování. Jedna z~odpovědí je poměrně rozsáhlejší a upozorňuje na povinnost přidávat při registraci profilový obrázek a na nemožnost vyhledávání podle data či textu v~sekci Hlasování, viz \ref{fig:dotaznik-zmeny}. Celkově uživatelé většinou hodnotí aplikaci nadprůměrně, viz \ref{fig:dotaznik-hodnoceni}

\begin{figure}[H]
    \centering
    \includegraphics[width=1\textwidth]{obrazky-figures/dotaznik-hodnoceni.png}
    \caption{Statistika hodnocení aplikace}
    \label{fig:dotaznik-hodnoceni}
\end{figure}

\begin{figure}[H]
    \centering
    \includegraphics[width=0.9\textwidth]{obrazky-figures/dotaznik-zmeny.png}
    \caption{Některé ze zajímavějších odpovědí v~dotazníku k~otázce co by uživatelé změnili}
    \label{fig:dotaznik-zmeny}
\end{figure}

\subsection{Závěry z~testování} 
Na základě výsledků testů produkční aplikace byly provedeny následující úpravy: 
\begin{itemize}
    \item Výsledky kalkulačky byly přesunuty z~uživatelského profilu do samostatné stránky s~odkazem z~hlavního menu. Díky tomu již nebude uživatel muset přemýšlet, kde má výsledky hledat. \item Tlačítko kalkulačky na úvodní straně nově odkazuje do sekce O~nás, kde je použití kalkulačky vysvětleno, místo přechodu přímo na sekci Hlasování.
    \item V~sekci O~nás byla funkce kalkulačky důkladněji popsána.
    \item Načítání sekce Hlasování bylo zrychleno
    \item Byla odstraněna povinnost při registraci nahrávat profilový obrázek
    \item Bylo přidáno vyhledávání podle data a textu v~názvu do sekce Hlasování. Vyhledání se uživateli zobrazí kliknutím na ikonu filtru. Na obrázku \ref{fig:aplikace-filter} je vyhledávací modul v~otevřeném stavu.
\end{itemize}


\begin{figure}
    \centering
    \includegraphics[width=1\textwidth]{obrazky-figures/aplikace-filter.png}
    \caption{Filtr hlasování přidaný na základě zpětné vazby}
    \label{fig:aplikace-filter}
\end{figure}

\chapter{Závěr}
\label{chap:zaver}

Cílem této práce bylo navrhnout a implementovat jednoduchý a efektivní nástroj, díky kterému si každý bude moci jednoduše porovnat svoje politické názory se skutečnými názory současných či bývalých členů Poslanecké sněmovny, který zároveň bude co nejvíce automatizován, aby úplně vymizela, či byla minimalizována nutnost lidského zásahu.
\par V~teoretické části byly rozebrány základní pojmy nutné pro pochopení fungování Poslanecké sněmovny a legislativního procesu. Následně byla vyjmenována již existující konkurenční řešení a popsán formát dostupných dat na kterých bude práce stavět. Poté jsem se zaměřil na popis použitých technologií. Teoretickou část uzavírá rozsáhlá kapitola o~návrhu aplikace, popisující jej od určení cílové skupiny a případů užití, přes vytváření modelů návrhů až po jejich skutečnou realizace ve formě grafické předlohy. V~závěru návrhové části jsou popsány některé teoretické aspekty důležité pro implementaci, například předpoklady k~výběru důležitých hlasování, či zpracování výsledků v~kalkulačce.
\par Na základě teoretické části poté vznikla funkční aplikace, jejíž nejdůležitější implementační detaily jsou popsány v~praktické části práce a která by po nasazení na skutečný produkční server, a doplnění potřebného nastavení, měla být schopna samostatného fungování pro veřejnost. Tento předpoklad pramení z~výsledků uživatelského testování, na základě kterého byly odstraněny všechny nalezené nedostatky. 
\par Zůstal zde však velký nevyužitý potenciál pro budoucí rozšíření. Například by bylo možné vylepšit moderování uživatelských komentářů, vylepšit filtrování v~sekci Hlasování, přidat nové možnosti pro registrované uživatele - například veřejný uživatelský profil a~sdílení výsledků aplikace na sociální sítě. Také by bylo možné přidat uživatelské hodnocení jednotlivých hlasování, podle kterých by mohlo jít následně hlasování řadit a mnoho dalšího. 
